\section{Ejemplo Numérico}

Un caso específico del experimento donde los rayos de luz se
caracteríza por una longitud de onda $\lambda = \qty{600}{\nano\m}$.
La separación entre las rendijas es $d = \qty{60}{\micro\m}$ y el
patrón se observa a una distancia $L_0 = \qty{1.7}{\m}$.

Hallar los dos primeros máximos y mínimos de la intensidad después de
la franja central. Hallar la intensidad promedio máxima dado 
$E_A = \qty{1}{\micro\V\per\m}$.

Por la aproximación mencionada anteriormente, al ser $\theta$ un ángulo
considerado pequeño:
\begin{enumerate}
    \item CIC
    \[
        \begin{derivation}
                \res{ \sin(\theta) \approx \tan(\theta) }\\
            \To\\
                \res{ d\,\tan(\theta) = n\,\lambda }\\
            \equiv\\
                \res{ d\,\frac{y_{n,\max}}{L_0} = n\,\lambda }\\
            \why[\To]{Los dos primeros máximos corresponden a $n=1,2$}\\
                \res{
                    y_{1,\max} = \frac{\lambda\,L_0}{d}\ 
                    \land\ 
                    y_{2,\max} = \frac{2\,\lambda\,L_0}{d}
                }\\
            \equiv\\
                \res{
                    y_{1,\max} = \qty{1.7e-2}{\m}\ 
                    \land\ 
                    y_{2,\max} = \qty{3.4e-2}{\m}
                }
        \end{derivation}
    \]
    \item CID
    Tomando la misma aproximación que antes:
    
    \begin{longderivation}
            \res{ d\,\tan(\theta) = \left(n + \frac{1}{2}\right)\lambda }\\
        \equiv\\
            \res{ d\,\frac{y_{n,\min}}{L_0} = \left(n + \frac{1}{2}\right)\lambda }\\
        \why[\To]{Los dos primeros mínimos corresponden a $n=0,1$}\\
            \res{
                y_{0,\min} = \qty{8.5e-3}{\m}\ 
                \land\ 
                y_{1,\min} = \qty{2.55e-2}{\m}
            }
    \end{longderivation}
    \item Intensidad promedio máxima
    Para hallar esto, hay que recordar que la intensidad promedio tiene
    la siguiente expresión:
    \[\left\langle S \right\rangle_T =
    \frac{2E_A^2\cos^2\left(\dfrac{k\delta}{2}\right)}{c\,\mu_0}
    = I_{n,\max}\cos^2\left(\dfrac{k\delta}{2}\right)\]

    De esto, la intensidad máxima para este caso en partícular es:
    \[I_{n,\max} = \qty{5.3052e-14}{\W\per\m^2}\]
\end{enumerate}

