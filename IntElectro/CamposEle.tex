\section{Campos Eléctricos y Magnéticos}

La energía que transporta una onda electromagnética, es dada por el
vector de Poynting $\vv{S}$, y su magnitud es la intensidad.

\[\vv{S} = \frac{\vv{E} \times \vv{B}}{\mu_0}\ ,\ 
\mu_0 = 4\pi\times 10^{-7}\si{\tesla\m\per\A}\]

Las ondas electromagnéticas tienen las siguientes propiedades:

\begin{enumerate}
    \item $\vv{E} \perp \vv{B}$ : el campo eléctrico es perpendicular
    al magnético.
    \item $\size{\vv{E}} = c\,\size{\vv{B}}\ ,\ 
    c: \text{velocidad de la luz}$ : Las magnitudes de ambos campos son
    proporcionales.
\end{enumerate}

Sea $\hat{\perp}$ la dirección perpendicular al plano formado por
$\vv{E}$ , $\vv{B}$. Considerando las propiedades mencionadas:

\[\vv{S} = \frac{\size{\vv{E}}^2}{c\,\mu_0}\hat{\perp} =
\frac{c\size{\vv{B}}^2}{\mu_0}\hat{\perp}\]

Como $\Psi(p) = E(p) = A(\delta)\sin(wt - \gamma)$, entonces, en este
caso:
\[\size{\vv{S}} = S = \frac{(A(\delta))^2\sin^2(wt-\gamma)}{c\,\mu_0}\]

Es de interés conocer el valor promedio de esta magnitud en el
intervalo de tiempo $T$, siendo este el periodo de la función armónica.

\begin{longderivation}
        \res{ \left\langle S \right\rangle_T }\\
    =\\
        \res{ \frac{1}{T}\Int{0,T} S(t) \diff{t} }\\
    =\\
        \res{ \frac{1}{T}\Int{0,T}
                \frac{(A(\delta))^2\sin^2(wt-\gamma)}{c\,\mu_0} }\\
    \why[=]{$T$ es el periodo de la función $\sin$}\\
        \res{ \frac{1}{T}\frac{(A(\delta))^2\,T}{2c\mu_0} }\\
    =\\
        \res{ \frac{4 E_A^2 \cos^2\left(\dfrac{k\delta}{2}\right)}{2c\mu_0} }\\
    =\\
        \res{ \frac{2E_A^2 \cos^2\left(\dfrac{k\delta}{2}\right)}{c\mu_0} }\\
    \why[=]{$I_{n,\text{máx}} = 2E_A^2/c\mu_0$}\\
        \res{ I_{n,\text{máx}}\cos^2\left(\dfrac{k\delta}{2}\right) } 
\end{longderivation}

Esta expresión, expresa la intensidad promedio en función de $\delta$,
que a su vez, es expresión de $\theta$, que por la diferencia en las
magnitudes del experimento, se puede tomar desde el punto medio de las
rendijas. En el diagrama del experimento, se puede graficar esta
intensidad, y ver que hay puntos del frente de onda en el que se
encuentra $p$, en donde nuestra percepción de la luz indica más o menos
brillo:

\begin{center}
    \begin{tikzpicture}
        \begin{axis}[
            axis line style={draw=none},
            tick style={draw=none},
            xtick=\empty,
            ytick=\empty,
            xmin=-1,xmax=5.5,
            ymin=-3,ymax=3,
        ]

        % ~~~~~~ Coordenadas ~~~~~~ %
        \coordinate (A) at (4,0);
        \coordinate (B) at (1,0);
        \coordinate (C) at (4,1.5);


        \draw(0,-2) -- (0,2) node[midway, above, sloped] {frente de onda};

        % ~~~~ Rendijas ~~~~%
        \draw[color=blue!50!black,draw opacity=2, -| ] (1,2) -- (1,1);
        \draw[color=blue!50!black,draw opacity=2, |-| ] (1,0.8) -- (1,-0.8);
        \draw[color=blue!50!black,draw opacity=2, -| ] (1,-2) -- (1,-1);


        % ~~~~ Rayos y emisores ~~~~%
        \draw[color=red!50!black,dashed, ->] (0,0.9) -- (1,0.9) --
            node[midway, yshift=6mm, xshift=-10mm, sloped] 
            {$\Psi_1$} (5,1.7) node[right] {$x$};
        \draw[color=yellow!40!black,dashed, ->] (0,-0.9) -- (1,-0.9) --
            node[midway, yshift=-6mm, xshift=-10mm, sloped]
            {$\Psi_2$} (5,2.3) 
            node[right,yshift=2mm] {$x'$};
        \filldraw (4,1.5) circle (0.3mm) node [below] {$p$};
        
        % ~~~~ Datos ~~~~ %
        \draw[|-|] (0.3,0.9) -- (0.3,-0.9) node[midway, fill=white]{$d$};
        \draw[ -> ] (0.8,0.6) -- (0.8,0.8);
        \draw[ -> ] (0.8,1.2) -- (0.8,1);
        \draw[ -> ] (0.8,0.9) to [bend left] (0.5, 1.5) node[above] {$a$};
        \draw[ -> ] (0.8,-0.6) -- (0.8,-0.8);
        \draw[ -> ] (0.8,-1.2) -- (0.8,-1);
        \draw[ -> ] (0.8,-0.9) to [bend right] (0.5, -1.5) node[below] {$a$};
        \draw[dashdotted, red] (1,0) -- (4,1.5);
        \draw[|-|] (1,-2.1) -- (4,-2.1) node[midway, fill=white] {$L_0$};
        \draw[dotted] (4,-2) -- (4,2);
        \draw[red] (1,0) -- (4.5,0) node[right] {$y=0$};
        \draw[dashdotted] (3.2,1) -- (3.2,2.2) node[left] {$I_{n,\text{máx}}$};
        \pic[draw, "$\theta$", angle eccentricity=1.5] {angle= A--B--C};

        \addplot[
            domain=-2:2,
            samples=100,
            trig format plots=rad,
            color=green!30!black,
            thick
        ]
        (
            {0.4*cos(6*x + pi) + 3.6},{x}
        );

        \end{axis}
    \end{tikzpicture}
\end{center}
