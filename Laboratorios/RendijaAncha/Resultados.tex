\section{Resultados}
En los resultados se puede ver que en general, con el $Y_0$ encontrado
tanto manualmente como con el programa se puede aproximar
$\lambda$ a $6.35 \times 10^{-4}$ que es el $\lambda$ teórico. Entre
los experimentos hechos, el $\lambda$ hallado con la rendija de
$\SI{0.16}{\mm}$ a partir de la medida manual de $2Y_0 = \SI{8}{\mm}$
es el resultado más alejado del teórico con un error del 10\%; y el
valor más aproximado es el hallado con los datos del programa con la
rendija de $\SI{0.08}{\mm}$, donde $2Y_0 = \SI{14.478}{\mm}$, que
resutó en una longitud de onda
$\lambda = 6.363 \times 10^{-4}\ \si{\mm}$, con un error de
aproximación del 0.16\%