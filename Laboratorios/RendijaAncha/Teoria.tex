\section{Introducción}

Cuando se tiene una rendija ancha, y se hace insidir un frente de onda plano,
    provoca que cada punto del frente de onda sea un emisor isotrópico de ondas,
    cuyo efecto neto es la superposición de todas las ondas que llegan a un punto p.

    Recordemos que la condición de interferencia constructiva está 
dada, bajo la condiciónde que $\theta \approx 0$, entonces:
Cuando $a\sin(\theta) \approx a\tan(\theta)$:
\[
    Y_{n,\max} = \frac{L_0\lambda}{a} \left(n + \frac{1}{2}\right)
\]

Para el experimento se hizo atravesar una luz lacer a traves de una rendija de cierto grosor hasta llegar a un lector que medía la intensidad de la luz en diversos ángulos. El lector estaba sincronizado a un programa que interpreta los datos y los grafica.

La medición que se hizo fue la de la intensidad de la luz al hacerla pasar por varias rendijas de diversos grosores en posiciones distintas, además de calcular la intensidad máxima tanto manualmente como con los datos del programa