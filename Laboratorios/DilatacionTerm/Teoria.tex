\section{Teoría}

En este experimento se busca replicar el fenómeno de dilatación térmica.
Este fenómeno consiste en una deformación que ocurre cuando un cuerpo
cambia de temperatura, es decir, recibe energía de tipo calor.

La ecuación que nos dice la cantidad de calor requerida para cambiar la
temperatura de un cuerpo es:
\[Q = m_V\,c_V\,\Updelta T\]
Donde $Q$ es calor, $m_V$ es la masa, $c_V$ es el calor específico y
$\Updelta T$ es el cambio en la temperatura. el subíndice $V$ especifica
en este caso el objeto en cuestión, que es una varilla.

Ya que, en este experimento, la varilla es calentada por un circuito. El
cual consta de un alambre el cual está enrollado a lo largo de toda la
varilla. Este alambre claramente se calienta, y el calor generado está dado
por la ecuación:
\[Q = I^2\,R\,\Updelta t\]
Donde $I$ es la corriente, $R$ es la resistencia y $\Updelta t$ es el tiempo.

Por último, en el montaje del experimento, se asegura que uno de los extremos
de la varilla se mantenga estático en el mismo punto, mientras que el otro,
está en contacto con un deformímetro, del cual se van a medir los datos de
interés.

El objetivo es poder obtener el coeficiente de dilatación lineal del material
de la varilla, el cual es empleado en la ecuación:
\[\Updelta L = \upalpha\,L_0\,\Updelta T\]
Donde $\Updelta L$ es el cambio de longitud de la varilla, $\upalpha$ es el
coeficiente de dilatación lineal, $L_0$ es la longitud inicial de la varilla
y $\Updelta T$ es el cambio en la temperatura.

La linealización con los datos de cambio en longitud y temperatura, nos
permitirán hallar una recta, de la cual su pendiente $p$, debe aproximarse a $\upalpha\,L_0$.