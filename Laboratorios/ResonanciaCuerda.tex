% preamble

\documentclass{article}

\usepackage{geometry}
\usepackage{fancyhdr}
\usepackage{graphicx}
\usepackage[hidelinks]{hyperref}
\usepackage{setspace}
\usepackage{mathtools}
\usepackage{amsfonts} % para símbolo de los naturales
\usepackage{logicDG}
\usepackage{SIunitx}
\usepackage{enumitem}
\usepackage[spanish, es-noshorthands, es-noquoting]{babel}

\pagestyle{fancy}
\fancyhf{}
\setlength{\headheight}{70.38103pt}
\rhead{\textit{David G., Laura R., Luisa R., María V.}}
\lhead{\includegraphics[width = 4cm]{\logo}}
\lfoot{Página \thepage}
\rfoot{Resonancia en una cuerda}
\renewcommand{\headrule}{\hbox to \headwidth{\color{rojoEci}\leaders\hrule height \headrulewidth\hfill}}
\renewcommand{\footrulewidth}{0.4pt}

\hyphenpenalty=10000

\newcommand{\logo}{"C:/Users/usuario/OneDrive/Documentos/U/logo-eci.png"}

\newcommand{\titlename}{Resonancia \\[10pt] En Una Cuerda}%
\renewcommand{\author}{{David Gómez, Laura Rincón, Luisa Rodríguez, María Vivas}}

\definecolor{rojoEci}{RGB}{225, 70, 49}

% Temporal func

\NewDocumentCommand{\partiald}{ o m m }%
    {
        \IfNoValueTF{#1}
        {
            \dfrac{{\partial} #2}{{\partial} #3}
        }{
            \dfrac{{\partial}^{#1} #2}{\partial {#3}^{#1}}
        }        
    }

% Enumi func

\renewcommand{\labelenumi}{(\Roman{enumi})}
\renewcommand{\labelenumii}{(\roman{enumii})}

\doublespacing
\begin{document}
\begin{titlepage}
    \begin{center}
        \vspace{1cm}

        \textbf{\Huge{\titlename}}

        \vspace{1.5cm}

        \textbf{\large{\author}}

        \vspace{3cm}

        \includegraphics[width=0.8\textwidth]{\logo}
        
        \vfill

        Física de Calor y Ondas

        Escuela Colombiana de Ingeniería Julio Garavito

        \today
    \end{center}
\end{titlepage}

\clearpage
\tableofcontents
\clearpage


\section{Fundamentación}

La fundamentación de este experimento se basa en la teoría de las ondas
mecánicas, siendo este el caso específico de una cuerda.

La obtención de la cinemática de esta onda en una cuerda, viene por la
solución a dos situaciones. Una es la de una cuerda ``infinita'' sobre
la que se produce una perturbación armónica. La otra es tomando el caso
en el que esta cuerda termina en una pared, la cual reacciona con la
fuerza que le ejerce la perturbación al llegar y se la devuelve a la
cuerda mientras una fuerza armónica actúa en el otro extremo.

El primer caso resulta en la siguiente ecuación diferencial:

\[
\begin{derivation}
        \res{ v^2 \partiald[2]{y}{x} = \partiald[2]{y}{t}}\\
    \why[\To]{ Solución armónica}\\
        \res{ y(x,t) = A \sin (k(x - vt) + \phi) }\\
    \why{ Tomando $w = kv$}\\
        \res{ y(x,t) = A \sin (kx - (wt - \phi)) }
\end{derivation}
\]

El segundo caso resulta en la combinación de ambas ondas,
la que se produce por la fuerza armónica y el reflejo desde la pared:

\[
\begin{derivation}
        \res{ y(x,t) = A \sin(kx - wt) + A \sin(kx + wt) }\\
    \why*{}\\
        \res{ y(x,t) = 2A \sin(kx)\cos(wt) }
\end{derivation} 
\]

En esta ultima ecuación, se puede ver que hay valores de $kx$ para los que
$y$ es nula en cualquier tiempo. Estas soluciones son 
$kx_n = n\pi\, ,n \in \mathbb{N}$

Por resultados más a fondo del primer caso, se podía llegar a que
$ k = \dfrac{2\pi}{\lambda}$, por lo que $x_n = \dfrac{n \lambda}{2}$

Al tener la condición $\dfrac{2 L_0}{n} = \lambda$, se puede ver que $\lambda$
se vuelve variable con $n$, y con los resultados de la anterior ecuación
diferencial, se puede ver que la frecuencia también, y 
$f_n = n\dfrac{v}{2L_0}$. Estas frecuencia son las que producen armónicos.

\section{Descripción}

Para el experimento, se montó una cuerda con un extremo a un altavoz, el cual
producía un movimiento vertical en forma de fuerza armónica a una frecuencia
ingresada. Al otro extremo, la cuerda colgaba de una polea, y tenía un peso
atado. Este peso producía una tensión, la cual se podía calcular fácilmente,
al igual que la densidad de masa por longitud, asumiendo uniformidad.

Mediante estos datos, era posible hallar la frecuencia del armónico
fundamental y por ende todos los demás, permitiéndonos visualizar
el efecto de resonancia en el sistema.

\section{Mediciones}

\begin{enumerate}
    \item Tensión sobre la cuerda
    
        Como se mencionó antes, en un extremo de la cuerda, colgaba una
        masa añadida, la cual constaba de un triángulo metálico y
        unas arandelas. Todo esto se pesó dando una masa de \qty{0.006949}{\kilogram}.

        Entonces, la tensión ejercida en la cuerda era de \qty{0.068167}{\newton}

    \item Densidad de masa por longitud
    
    La densidad de masa por longitud no es más que la razón entre masa y longitud.
    Ya que se asumió uniformidad de esta densidad a lo largo de toda la cuerda,
    se tomó la longitud total de la cuerda (\qty{1.41}{\metre}) y su masa
    (\qty{0.00007}{\kilogram}), dando como resultado una densidad de masa
    $\mu = \qty[per-mode=symbol]{4.96454e-05}{\kilogram\per\metre}$

    \item Velocidad de onda
    
    La velocidad es la raíz del cociente entre la tensión y la densidad
    de masa. 
    
    Entonces $v = \qty[per-mode=symbol]{37.055040}{\metre\per\s}$

    \item frecuencias de los armónicos
    
    Se sabe que el primer armónico se encuentra con $f_1$ y los demás
    armónicos son múltiplos de este. Para el experimento, hallamos los
    tres primeros armónicos:

    \begin{enumerate}
        \item $f_1 = \qty{13.14008521}{\hertz}$
        \item $f_2 = \qty{26.28017042}{\hertz}$
        \item $f_3 = \qty{39.42025563}{\hertz}$
    \end{enumerate}
\end{enumerate}


\section{resultados}

A la hora de usar estas frecuencias en la cuerda, se observó con bastante
precisión el fenómeno de los armónicos, con los nodos bastante bien
definidos en cada una de las frecuencias.

\begin{center}
    \includegraphics[width=0.6\textwidth]{"C:/Users/usuario/Downloads/temp/ImagenLab.JPG"}
\end{center}

En la foto, se muestra uno de los nodos generados en la cuerdas tras aplicar
alguna de las frecuencias armónicas.
\end{document}
