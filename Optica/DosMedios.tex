\section{Óptica en dos medios}

Considerando una frontera con forma de arco, un objeto fuera de este
medio (el lente), y un observador que se encuentra dentro del medio del
lente. Para analizar esta situación consideramos lo siguiente:
\begin{itemize}
    \item El eje óptico:
    Es una traza que divide, en este caso al lente, por la mitad,desde
    el punto de vista que se analiza.
    \item Los índices de refracción del medio $1$ Y $2$:
    En este caso, tomamos $n_1 < n_2$.
    \item La ley de Snell:
    Es un resultado obtenido de considerar el cambio de dirección de
    un rayo de luz al cambiar de medio, este resulta en la igualdad
    \[n_1 \sin(\phi_1) = n_2 \sin(\phi_2)\]
    Donde los $n$ son índices de refracción y los ángulos son medidos
    desde la recta perpendicular al punto en el que se cambia de medio.
\end{itemize}

Para continuar con el análisis, consideramos dos rayos provenientes de
un objeto $s$ al otro lado del lente.

\input{Optica/Grafica1.tex}

El rayo amarillo (1) y el dado por el eje óptico (2) serán los dos rayos a
considerar para el análisis. Para hacer uso de la ley de Snell, se toma
la tangente al lente en donde el rayo 1 incide al medio y su respectiva
perpendicular.
\[n_1 \sin(\theta_i) = n_2 \sin(\theta_f)\]

Bajo la suposición de que todos los ángulos mostrados son pequeños, se
hacen las siguientes aproximaciones:
\begin{align*}
    \theta_i,\theta_f,\alpha, \beta, \gamma &\approx 0\\
    \sin(\theta_i,\theta_f,\alpha, \beta, \gamma) 
        &\approx \theta_i,\theta_f,\alpha, \beta, \gamma\\
    \tan(\theta_i,\theta_f,\alpha, \beta, \gamma)
        &\approx \theta_i,\theta_f,\alpha, \beta, \gamma
\end{align*}

Ya con estas aproximacines, se procede con el análisis:

\begin{longderivation}
        \res{
            \begin{sisEq}
                \alpha + \beta + o      &= \pi\\
                \theta_i + o            &= \pi\\
                \gamma + C + \theta_f   &= \pi\\
                \beta + C               &= \pi
            \end{sisEq}
        }\\
    \To\\
        \res{
            \begin{sisEq}
                \alpha + \beta      &= \theta_i\\
                \gamma + \theta_f   &= \beta
            \end{sisEq}
        }\\
    \why[\To]{Ley de Snell y aproximaciones: $n_1\theta_i = n_2\theta_f$}\\
        \res{ \gamma + \frac{n_1}{n_2}\theta_i = \beta}\\
    \To\\
        \res{ \theta_i = \frac{n_2}{n_1}(\beta - \gamma) = \alpha + \beta}\\
    \why[\To]{Aproximaciones}\\ 
        \res{ \frac{n_2}{n_1}\left(\frac{h}{R} - \frac{h}{q}\right)
            = \frac{h}{p} + \frac{h}{R} }\\
    \equiv\\
        \res{ \frac{n_2 - n_1}{R} = \frac{n_1}{p} + \frac{n_2}{q} }
\end{longderivation}

Ahora, se introducirá una convención para este tipo de situaciones:
dado un objeto o emisor de luz, un cambio de medio y un receptor
(observador). Es  llamado ``delante'' el espacio entre el objeto y el
cambio de medio, y ``detrás'' el espacio entre el cambio de medio y el
receptor. A su vez, se asignan signos a las distancias que se denotaron
como $p$ y $q$ según esta conveniencia:

\begin{center}
    \begin{tabular}{ | c | c | c | } 
        \hline
        Variable    & $+$            & $-$\\ 
        \hline
        $p$         & delante        & detrás \\ 
        \hline
        $R$         & centro detrás  & centro delante\\ 
        \hline
      \end{tabular}
\end{center}
