\section{Óptica en dos medios}

Considerando una frontera con forma de arco, un objeto fuera de este
medio (el lente), y un observador que se encuentra dentro del medio del
lente. Para analizar esta situación consideramos lo siguiente:
\begin{itemize}
    \item El eje óptico:
    Es una traza que divide, en este caso al lente, por la mitad,desde
    el punto de vista que se analiza.
    \item Los índices de refracción del medio $1$ Y $2$:
    En este caso, tomamos $n_1 < n_2$.
    \item La ley de Snell:
    Es un resultado obtenido de considerar el cambio de dirección de
    un rayo de luz al cambiar de medio, este resulta en la igualdad
    \[n_1 \sin(\phi_1) = n_2 \sin(\phi_2)\]
    Donde los $n$ son índices de refracción y los ángulos son medidos
    desde la recta perpendicular al punto en el que se cambia de medio.
\end{itemize}

Para continuar con el análisis, consideramos dos rayos provenientes de
un objeto $s$ al otro lado del lente.

\begin{center}
    \begin{tikzpicture}[scale=2]
        \begin{axis}[
            xmin=-3,
            xmax=3,
            axis equal,
            hide axis,
        ]
            \addplot[
                domain=0.7*pi:1.3*pi,
                trig format plots=rad,
                line width=0.01mm,
                yellow!60!black,
            ](
                {cos(x)},
                {sin(x)}
            );
            \addplot[
                domain=-1:-0.75,
                line width = 0.005mm,
            ]{
                1.7320581*(x+0.866025404) + 0.5
            };
            \addplot[
                domain=-1.5:0,
                dashed,
                line width=0.01mm,
                opacity=0.5,
            ]{
                -0.577350269*(x+0.866025404) + 0.5
            };
            \draw[line width=0.1mm, dotted] (-1,0.2) -- (-1,-1);

            % ~~~~ Coordenadas ~~~~ %
            \coordinate (qi1) at (-1.3,0.750555350);
            \coordinate (qi2) at (-0.866025,0.5);
            \coordinate (qi3) at (-2,0);

            \coordinate (qf1) at (0,0);
            \coordinate (qf2) at (-0.866025,0.5);
            \coordinate (qf3) at (2,-0.2);

            \coordinate (a1) at (-1,0);
            \coordinate (a2) at (-2,0);
            \coordinate (a3) at (-0.866025,0.5);

            \coordinate (o1) at (-2,0);
            \coordinate (o2) at (-0.866025,0.5);
            \coordinate (o3) at (0,0);

            \coordinate (g2) at (1.18113560,0);


            \draw[orange] (-2,0) -- (-0.866025,0.5);
            \draw[orange] (-0.866025,0.5) -- (2,-0.2);
            \draw[loosely dashed] (-2.1,0) -- (2.1,0) node [above,scale=0.5] {Eje Óptico};
            \draw[red!30!black,|-|, line width=0.1mm] (-0.7,0.5) -- (-0.7,0);  

            % ~~~~~ Ángulos ~~~~ %
            \pic[
                draw,
                "\resizebox{2mm}{2mm}{$\theta_i$}",
                angle eccentricity=1.5]
                {angle = qi1--qi2--qi3};
            \pic[
                draw,
                "\resizebox{1.5mm}{1.2mm}{$\theta_f$}",
                angle eccentricity=1.5] {angle = qf1--qf2--qf3};
            \filldraw (0,0) circle (0.1mm) node[below] {\resizebox{2mm}{2mm}{$C$}};
            \draw[ ->,
            red!30!black] (0,0) -- (-0.866025,-0.5) node[xshift=2mm]
            {\resizebox{1.5mm}{1.5mm}{$R$}};
            \pic[
                draw,
                "\resizebox{1.5mm}{1.2mm}{$\alpha$}",
                angle eccentricity=1.5] {angle = a1--a2--a3};
            \pic[
                draw,
                "\resizebox{1.2mm}{1.2mm}{$o$}",
                angle eccentricity=1.7, angle radius = 1.5mm] {angle = o1--o2--o3};
            \pic[
                draw,
                angle radius = 1.3mm] {angle = qf3--qf1--qf2};
            \pic[
                draw,
                "\resizebox{1.5mm}{1.5mm}{$\beta$}",
                angle eccentricity=1.5,
                angle radius=4.3mm] {angle = a3--qf1--a1};
            \pic[
                draw,
                "\resizebox{1.2mm}{1.2mm}{$\gamma$}",
                angle eccentricity=1.2,
                angle radius=8.2mm] {angle = qf2--g2--qf1};

            \draw[|-|] (-2,-0.6) -- (-0.992,-0.6) node[midway,fill=white]
            {\resizebox{1.5mm}{1.5mm}{$p$}};
            \filldraw (1.18113560,0) circle (0.2mm) node[above]
            {\resizebox{1.5mm}{1.5mm}{$I$}};
            \draw[|-|] (-1.005,-0.7) -- (1.18113560,-0.7) node[midway, fill=white]
            {\resizebox{1.5mm}{1.5mm}{$q$}};

            \draw (-1,1) node {\resizebox{3.5mm}{1.5mm}{$n_1$}};
            \draw (0,1) node[yellow!60!black] {\resizebox{3.5mm}{1.5mm}{$n_2$}};
            \draw[ ->,red!30!black] (-0.7,0.25) to[bend right] (-0.5,0.6) node[above]
            {\resizebox{1.5mm}{1.75mm}{$h$}};
            \filldraw[orange] (-2,0) circle (0.5mm) node[below]
            {\resizebox{1.5mm}{1.5mm}{$s$}};
            
        \end{axis}
    \end{tikzpicture}
\end{center}

El rayo amarillo (1) y el dado por el eje óptico (2) serán los dos rayos a
considerar para el análisis. Para hacer uso de la ley de Snell, se toma
la tangente al lente en donde el rayo 1 incide al medio y su respectiva
perpendicular.
\[n_1 \sin(\theta_i) = n_2 \sin(\theta_f)\]

Bajo la suposición de que todos los ángulos mostrados son pequeños, se
hacen las siguientes aproximaciones:
\begin{align*}
    \theta_i,\theta_f,\alpha, \beta, \gamma &\approx 0\\
    \sin(\theta_i,\theta_f,\alpha, \beta, \gamma) 
        &\approx \theta_i,\theta_f,\alpha, \beta, \gamma\\
    \tan(\theta_i,\theta_f,\alpha, \beta, \gamma)
        &\approx \theta_i,\theta_f,\alpha, \beta, \gamma
\end{align*}

Ya con estas aproximacines, se procede con el análisis:

\begin{longderivation}
        \res{
            \begin{sisEq}
                \alpha + \beta + o      &= \pi\\
                \theta_i + o            &= \pi\\
                \gamma + C + \theta_f   &= \pi\\
                \beta + C               &= \pi
            \end{sisEq}
        }\\
    \To\\
        \res{
            \begin{sisEq}
                \alpha + \beta      &= \theta_i\\
                \gamma + \theta_f   &= \beta
            \end{sisEq}
        }\\
    \why[\To]{Ley de Snell y aproximaciones: $n_1\theta_i = n_2\theta_f$}\\
        \res{ \gamma + \frac{n_1}{n_2}\theta_i = \beta}\\
    \To\\
        \res{ \theta_i = \frac{n_2}{n_1}(\beta - \gamma) = \alpha + \beta}\\
    \why[\To]{Aproximaciones}\\ 
        \res{ \frac{n_2}{n_1}\left(\frac{h}{R} - \frac{h}{q}\right)
            = \frac{h}{p} + \frac{h}{R} }\\
    \equiv\\
        \res{ \frac{n_2 - n_1}{R} = \frac{n_1}{p} + \frac{n_2}{q} }
\end{longderivation}

Ahora, se introducirá una convención para este tipo de situaciones:
dado un objeto o emisor de luz, un cambio de medio y un receptor
(observador). Es  llamado ``delante'' el espacio entre el objeto y el
cambio de medio, y ``detrás'' el espacio entre el cambio de medio y el
receptor. A su vez, se asignan signos a las distancias que se denotaron
como $p$ y $q$ según esta conveniencia:

\begin{center}
    \begin{tabular}{ | c | c | c | } 
        \hline
        Variable    & $+$            & $-$\\ 
        \hline
        $p$         & delante        & detrás \\ 
        \hline
        $R$         & centro detrás  & centro delante\\ 
        \hline
      \end{tabular}
\end{center}
