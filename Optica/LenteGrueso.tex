\section{Lente Grueso}

\begin{center}
    \begin{tikzpicture}[scale=2]
        \begin{axis}[
            xmin=-3,
            xmax=3,
            axis equal,
            hide axis,
        ]
            \addplot[
                domain=0.7*pi:1.3*pi,
                trig format plots=rad,
                line width=0.01mm,
                yellow!60!black,
            ](
                {cos(x)},
                {sin(x)}
            );
            \addplot[
                domain=-0.3*pi:0.3*pi,
                trig format plots=rad,
                line width=0.01mm,
                yellow!60!black,
            ](
                {cos(x)-1.17557050},
                {sin(x)}
            );
            \addplot[
                domain=-1:-0.75,
                line width = 0.005mm,
            ]{
                1.7320581*(x+0.866025404) + 0.5
            };
            \addplot[
                domain=-1.5:0,
                dashed,
                line width=0.01mm,
                opacity=0.5,
            ]{
                -0.577350269*(x+0.866025404) + 0.5
            };
            \draw[line width=0.1mm, dotted] (-1,0.2) -- (-1,-2);
            \draw[line width=0.1mm, dotted] (-0.17557050,0.2) -- (-0.17557050,-2);
            \draw[|-|,red!30!black]  (-1,-2) -- (-0.17557050,-2)
            node[midway, fill=white]
            {\resizebox{1.5mm}{1.5mm}{$\varepsilon$}};
            \draw[|-|]  (-0.17557050,-1.5) -- (1.18113560,-1.5)
            node[midway, fill=white]
            {\resizebox{3mm}{1.5mm}{$p_2$}};
            \draw[ ->, blue!30!black] (-0.224513988,-0.309016994) -- (-1.17557050,0)
            node[yshift=-2mm] {\resizebox{1.8mm}{1.2mm}{$R_2$}};

            % ~~~~ Coordenadas ~~~~ %
            \coordinate (qi1) at (-1.3,0.750555350);
            \coordinate (qi2) at (-0.866025,0.5);
            \coordinate (qi3) at (-2,0);

            \coordinate (qf1) at (0,0);
            \coordinate (qf2) at (-0.866025,0.5);
            \coordinate (qf3) at (2,-0.2);

            \coordinate (a1) at (-1,0);
            \coordinate (a2) at (-2,0);
            \coordinate (a3) at (-0.866025,0.5);

            \coordinate (o1) at (-2,0);
            \coordinate (o2) at (-0.866025,0.5);
            \coordinate (o3) at (0,0);

            \coordinate (g2) at (1.18113560,0);


            \draw[orange] (-2,0) -- (-0.866025,0.5);
            \draw[orange] (-0.866025,0.5) -- (2,-0.2);
            \draw[loosely dashed] (-2.1,0) -- (2.1,0) node [above,scale=0.5] {Eje Óptico};
            \draw[red!30!black,|-|, line width=0.1mm] (-0.7,0.5) -- (-0.7,0);  

            % ~~~~~ Ángulos ~~~~ %
            \pic[
                draw,
                "\resizebox{2mm}{2mm}{$\theta_i$}",
                angle eccentricity=1.5]
                {angle = qi1--qi2--qi3};
            \pic[
                draw,
                "\resizebox{1.5mm}{1.2mm}{$\theta_f$}",
                angle eccentricity=1.5] {angle = qf1--qf2--qf3};
            \filldraw (0,0) circle (0.1mm) node[below] {\resizebox{2mm}{2mm}{$C$}};
            \draw[ ->,
            red!30!black] (0,0) -- (-0.866025,-0.5) node[xshift=2.5mm]
            {\resizebox{2.5mm}{1.5mm}{$R_1$}};
            \pic[
                draw,
                "\resizebox{1.5mm}{1.2mm}{$\alpha$}",
                angle eccentricity=1.5] {angle = a1--a2--a3};
            \pic[
                draw,
                "\resizebox{1.2mm}{1.2mm}{$o$}",
                angle eccentricity=1.7, angle radius = 1.5mm] {angle = o1--o2--o3};
            \pic[
                draw,
                angle radius = 1.3mm] {angle = qf3--qf1--qf2};
            \pic[
                draw,
                "\resizebox{1.5mm}{1.5mm}{$\beta$}",
                angle eccentricity=1.5,
                angle radius=4.3mm] {angle = a3--qf1--a1};
            \pic[
                draw,
                "\resizebox{1.2mm}{1.2mm}{$\gamma$}",
                angle eccentricity=1.2,
                angle radius=8.2mm] {angle = qf2--g2--qf1};

            \draw[|-|] (-2,-0.6) -- (-0.992,-0.6) node[midway,fill=white]
            {\resizebox{3mm}{1.5mm}{$p_1$}};
            \filldraw (1.18113560,0) circle (0.2mm) node[above]
            {\resizebox{1.5mm}{1.5mm}{$I$}};
            \draw[|-|] (-1.005,-0.7) -- (1.18113560,-0.7) node[midway, fill=white]
            {\resizebox{3mm}{1.5mm}{$q_1$}};

            \draw (-1,1) node {\resizebox{3.5mm}{1.5mm}{$n_1$}};
            \draw (0,1) node[yellow!60!black] {\resizebox{3.5mm}{1.5mm}{$n_2$}};
            \draw[ ->,red!30!black] (-0.7,0.25) to[bend right] (-0.5,0.8) node[yshift=1mm]
            {\resizebox{1.5mm}{1.75mm}{$h$}};
            \filldraw[orange] (-2,0) circle (0.5mm) node[below]
            {\resizebox{1.5mm}{1.5mm}{$s$}};
            
        \end{axis}
    \end{tikzpicture}
\end{center}

En este caso, el medio dos, tiene un final antes de llegar al receptor.
Esto no es otra cosa que la combinación de dos situaciones iguales a
la vista anteriormente. La luz pasa por el primer cambio de medio y
tiene una respectiva proyección, esta viene a ser el objeto que tomará
el otro lado del lente, sin embargo, hay que recordar que la luz
sigue proviniendo de la posición real del objeto.

Esto es, el delante y detrás de la primera (i) superficie, se mantiene
igual que antes. En el caso de la superficie dos (ii), el delante está
en el detrás de (i), y el detrás está entre (ii) y el receptor.

Ya que este problema no es más que tomar los resultados del anterior y
añadir el mismo proceso con el lente (ii). Contamos con, precisamente,
las ecuaciones de la situación anterior. Nótese que este resultado,
con las conveniencias del delante y detrás, se pueden aplicar con el
lente (ii).

\begin{longderivation}
        \res{
            \begin{sisEq}
                \frac{n_2 - n_1}{R_1}   &= \frac{n_1}{p_1} + \frac{n_2}{q_1}\\[10pt]
                \frac{n_1 - n_2}{R_2}   &= \frac{n_2}{p_2} + \frac{n_1}{q_2}\\
                        p_2             &=      \varepsilon - q_1
            \end{sisEq}
        }\\
    \To\\
        \res{
            \begin{sisEq}
                \frac{n_2 - n_1}{R_1}   &= \frac{n_1}{p_1} + \frac{n_2}{q_1}\\[10pt]
                \frac{n_1 - n_2}{R_2}   &= \frac{n_2}{\varepsilon - q_1} + \frac{n_1}{q_2}
            \end{sisEq}
        }\\
    \equiv\\
            \res{
                \begin{sisEq}
                    \frac{n_2 - n_1}{R_1}           &= \frac{n_1}{p_1} + \frac{n_2}{q_1}\\[10pt]
                    \frac{n_2}{q_1 - \varepsilon}   &= \frac{n_1}{q_2} - \frac{n_1 - n_2}{R_2}
                \end{sisEq}
            }\\
    \why[\To]{ Considerando el ancho del lente como pequeño: $p_2 \approx -q1$}\\
            \res{\frac{n_2 - n_1}{R_1} = 
                \frac{n_1}{p_1} + \frac{n_1}{q_2} - \frac{n_1 - n_2}{R_2} }\\
    \equiv\\
            \res{ (n_2 - n_1)\left(\frac{1}{R_1} - \frac{1}{R_2}\right) 
                = \frac{n_1}{p_1} + \frac{n_1}{q_2}}
\end{longderivation}

Esta última igualdad es llamada ecuacuión del fabricante de lentes, y
resulta ser igual al inverso de la distancia focal de la lente delgada $f$.
Esta se mide desde el lente.