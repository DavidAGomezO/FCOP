\section{Ejemplo Numérico}

Considerando ahora un objeto puesto en otro punto del espacio, no
necesariamente atravesado por el eje óptico, la estrategia es tomar dos
rayos: uno debe ser paralelo al eje óptico, y el seguno debe pasar por
el centro del lente. Considerando el caso de un sistema con dos lentes,
donde uno (1) es convergente (los rayos se dirigen al eje óptico), y el
otro (2) es divergente (los rayos se alejan del eje óptico); puestos en ese
orden.

Sean $p_c = \qty{4}{\cm}$ la distancia del objeto al lente (1),
la distancia focal de (1) $f_c=\qty{1.5}{\cm}$, la distancia focal de
(2) $f_d = \qty{-1.5}{\cm}$, y la separación entre los lentes
$s = \qty{5}{\cm}$.

Aproximando el índice de refracción del aire al del vacío, $n_1 = 1$.

Lo primero es determinar la proyección del objeto tras el lente (1).
La distancia a la que se proyecta será llamada $q_c$.

\begin{longderivation}
        \res{\frac{1}{p_c} + \frac{1}{q_c} = \frac{1}{f}}\\
    \equiv\\
        \res{ q_c = \frac{f_c\,p_c}{p_c - f_c} }\\
    \equiv\\
        \res{ q_c = \qty{2.4}{\cm} }
\end{longderivation}

La distancia de la proyección de (1) a (2) será llamada $p_d$. Por lo
descrito anteriormente en el enunciado $p_d = s - q_c = \qty{2.6}{\cm}$.

Análogamente, $q_d$ será la distancia a la que la imagen se proyecta desde (2):
\begin{longderivation}
        \res{\frac{1}{p_d} + \frac{1}{q_d} = \frac{1}{f_d}}\\
    \equiv\\
        \res{ q_d = \frac{f_d\,p_d}{p_d - f_d} }\\
    \equiv\\
        \res{ q_d = \qty{-0.95122}{\cm} }
\end{longderivation}