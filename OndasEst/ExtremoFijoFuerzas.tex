\subsection{Análisis de fuerzas}

Por tercera ley de Newton, se tiene que, la pared, refleja la
información de la perturbación que le llega del otro extremo de la
cuerda. Esta reacción es, por definición, igual pero en sentido opuesto.

\begin{center}
    \begin{tikzpicture}
        \begin{axis}[
            axis line style={draw=none},
            tick style={draw=none},
            xtick=\empty,
            ytick=\empty,
            xmin=0,xmax=2,
            ymin=-2,ymax=2,
        ]
        
        \draw[thick] (2,-1) -- (1.5,-1) -- (1.5,1) -- (2,1);
        \addplot[smooth] coordinates{(0,0) (0.5,0.8) (1,1) (1.5,0.4)};
        \draw[ -> ] (1.5,0.4) -- (1,1.2) 
            node [midway, yshift=5mm] {$-F\hat{x^*}$};
        \draw[ -> , dotted] (1.5,0.4) -- (2,-0.4) 
            node[midway, yshift=3mm, xshift=2mm] {$F\hat{x^*}$};
        \end{axis}
    \end{tikzpicture}
\end{center}

La onda incidente, por lo visto en la cinemática de Ondas Mecánicas, es
de la forma
\[y = A\sin\left(kx - wt\right)\]

Por transformación de funciones, se puede ver que el efecto opuesto,
tiene la forma
\[y = A\sin(kx + wt)\]

Esta última función tiene sentido una vez la perturbación haya vuelto
al punto inicial. Esto es, cuando la perturbación haya recorrido $2L_0$.
Esto, tomando en cuenta que la perturbación tiene una velocidad $v$, es
lo mismo a que la función sea válida en $t = \dfrac{2L_0}{v}$.
Si tomamos ahora un nuevo punto de referencia temporal, desde que la
cuerda es afectada por la onda reflejada, $t'=\dfrac{2L_0}{v}$.
Entonces, la cinemática en la cuerda estará dado por:
\[
    \begin{derivation}
            \res{ y(x,t') = A\sin(kx - wt') + A\sin(kx+wt') }\\
        \equiv\\
            \res{ y(x,t') = 2A\sin(kx)\cos(wt') }
    \end{derivation}
\]

De este resultado se puede observar que hay valores de $t'$ y $x$ para
los que la función se iguala a $0$ y estos no dependen del otro. Es
decir, hay puntos de la cuerda que se mantendrán estáticos sin importar
el valor de $t'$, y hay momentos en los que todos los puntos de la
cuerda pasan simultáneamente por $y=0$.

Los valores de $x$ mencionados anteriormente, son a su vez dependientes
de un natural $n$, por lo que se llamarán $x_n$

\[
    \begin{derivation}
            \res{ y(x,t') = 0 }\\
        \equiv\\
            \res{ kx = n\pi \ ;\ n \in \mathbb{N}}\\
        \equiv\\
            \res{x_n = \frac{1}{k} n\pi\ ;\ n \in \mathbb{N}}\\
        \why{Por definición de longitud de onda}\\
            \res{ x_n = n \frac{\lambda}{2} }
    \end{derivation}
\]

Por otra parte, una de las condiciones dadas era que, a lo largo de la
cuerda, hubiera una cantidad entera de semi-longitudes de onda. Esto es,
$L_0 = N\,\dfrac{\lambda}{2}$ para algún $N \in \mathbb{N}$. De esta
igualdad, se puede expresar $\lambda$ como dependiente de $N$:
\[\lambda_n = \frac{2L_0}{n}\]
Ahora, de la frecuencia $f$ de esta onda, se tiene que $f\lambda = v$,
por lo que la frecuencia también depende de un número natural:
\[f_n = \frac{v}{\lambda_n} = \frac{n\,v}{2L_0}\]

De esto se concluye que esta cinemática solo se produce cuando con las
frecuencias dadas por un natural $n$.

\subsection{Ejemplo Numérico}

Una cuerda con con un extremo fijo, de longitud $L_0 = \qty{2}{\m}$,
peso total de $m = \qty{0.07}{\kg}$ y con una tensión constante
$T_n = \qty{0.5}{\N}$. Hallar las frecuencias de resonancia como
función de $\mathbb{N}$

Como la densidad de masa lineal es constante, lo es en especial para
toda la longitud y la masa total: 
$\mu = \dfrac{m}{L_0} = \qty{0.035}{\kg\per\m}$

Así, la velocidad de propagación de la onda es:
$v = \sqrt{\dfrac{T_n}{\mu}} = \qty{3.7796}{\m\per\s}$

Por último, se tiene que $f_n = \dfrac{n\,v}{2L_0}$. Reemplazando:
\[f_n = 9.4491\,n\]
