\section{Cuerda Con Extremo Fijo}

Considerando ahora el caso de una cuerda, con densidad linearl de masa
constante, tensión constante, y en la que uno de sus extremos es fijo,
o está en una pared, como en este caso:

\begin{center}
    \begin{tikzpicture}
        % ~~~~~~~ coordinates ~~~~~~~~%
        \coordinate (A) at (0,0);
        \coordinate (B) at (5,0);

        % ~~~~~~~~ actual thing ~~~~~~~~%
        \draw[pattern=north west lines] (5,-0.3) rectangle (5.5,0.3);
        \draw (0,0) -- (5,0);
        \draw[|-|] ([yshift=5mm] A) -- ([yshift=5mm] B) node[midway, fill=white] {$L_0$};
    \end{tikzpicture}
\end{center}

Debido a esto, hay condiciones de contorno y condiciones iniciales para
la ecuación de onda, lo que resultará en otra solución específica.

\begin{gather*}
    y(0,0) = 0\\
    y(L_0,t) = 0\\
    \pdv{y}{t}(0,0) = -A\,w\\
    \pdv{y}{x}(L_0,0) = 0
\end{gather*}

Resulta que estas condiciones son válidas si la perturbación en la
cuerda inicia justo en el instante $t=0$. Además, si la perturbación
llega al extremo de la pared en un tiempo $t = \dfrac{L_0}{v}$, se
produce una cantidad entera de semi-longitudes de onda a lo largo de
la cuerda.
