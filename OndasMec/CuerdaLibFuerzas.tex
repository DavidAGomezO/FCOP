\subsection{Análisis de fuerzas}

Por segunda ley de Newton:

\begin{longderivation}
        \res{ \sum \vv{T_{n,i}} = \Delta m\vv{a} }\\
    \equiv\\
        \res{ \vv{T_{n,1}} + \vv{T_{n,2}} = \Delta m\vv{a} }\\
    \why{ Asumimos que la tensión es constante en toda la cuerda }\\
        \res{ \vv{T_n}\cos(\theta_1)(-\hat{x})
            + \vv{T_n}\sin(\theta_1)(-\hat{y})
            + \vv{T_n}\cos(\theta_2)(\hat{x})
            + \vv{T_n}\sin(\theta_2)(\hat{y})}\\
    \why{ Las partículas no tienen movimiento longitudinal }\\
        \res{\vv{T_n}\sin(\theta_1)(-\hat{y})
            + \vv{T_n}\sin(\theta_2)(\hat{y})}\\
    \why{ Se asume que la deformación es pequeña }\\
        \res{ \vv{T_n}\tan(\theta_1)(-\hat{y})
        + \vv{T_n}\tan(\theta_2)(\hat{y}) }\\
    \why{ la tangente es la pendiente de la recta tangente, $\Delta m = \mu \Delta x$ }\\
        \res{ T_n\left(
            \frac{\odv{y}{x}(x + \Delta x) - \odv{y}{x}(x)}{\Delta x}
            \right) = \mu a }\\
    \To\\
        \res{ T_n \Lim{\Delta x}{0} \left(
            \frac{\odv{y}{x}(x + \Delta x) - \odv{y}{x}(x)}{\Delta x}
            \right) =  \mu a}\\
    \equiv\\
        \res{ T_n \odv[order=2]{y}{x} = \mu a }\\
    \equiv\\
        \res{ T_n \odv[order=2]{y}{x} = \mu \odv[order=2]{y}{t} }\\
    \why{ Notación }\\
        \res{ T_n \pdv[order=2]{y}{x} = \mu \pdv[order=2]{y}{t} }\\
    \equiv\\
        \res{ \frac{T_n}{\mu} \pdv[order=2]{y}{x} = \pdv[order=2]{y}{t} }\\
    \equiv\\
        \res{ v^2 \pdv[order=2]{y}{x} = \pdv[order=2]{y}{t} }
\end{longderivation}

Primero hace falta comprobar que las unidades de la velocidad sean correctas:

\[
    \begin{derivation}
            \res{ \left[\frac{T_n}{\mu}\right] }\\
        =\\
            \res{ \frac{\si{\N}}{\si{\kg\per\m}} }\\
        =\\
            \res{ \frac{(\si{\kg})(\si{\meter\per\second^2})}{\si{\kg\per\m}} }\\
        =\\
            \res{ \left(\frac{\si{\m}}{\si{\s}}\right)^2 }
    \end{derivation}
\]

Una de las soluciones de la ecuación diferencial, es llamada
``solución armónica'', la cual resulta en una función que se puede
expresar como función de una sola variable:
\[y(x, t) = y^*(x - vt) = A\sin(k(x-vt) + \phi)\]

La razón de esta expresión de una variable, viene del hecho de que la
perturbación va a viajar a una velocidad $v$, y por ende, en tras un
tiempo $t$, la función en $XY$ se va a repetir.