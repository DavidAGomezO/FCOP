\subsection{Análisis de energía}

Ya que la solución armónica produce comportamientos de este estilo,
se hará el análisis en el intervalo para $t$ correspondiente al
periodo temporal $T$ de la función para un $x$ fijo. De igual forma,
en el intervalo para $x$ correspondiente al periodo espacial $\lambda$
de la función para un $t$ fijo.

Se puede ver que el comportamiento de la cuerda en un instante de
tiempo es el mismo que tiene una partícula de la cuerda en un intervalo
de tiempo. Sin embargo, la única relación certera que se tiene
(numéricamente) entre ambas situaciones, son precisamente los
intervalos mencionados anteriormente. Esto es, la energía a lo largo de
un trozo de la cuerda de longitud $\lambda$ es la misma que tiene una
única partícula de la cuerda a lo largo de un tiempo $T$.

La energía mecánica a lo largo de un tramo de longitud $\lambda$ es:

\[E_{\lambda} = E_{c,\lambda} + E_{p,\lambda}\]

La igualdad entre ambos casos, se puede ver, corresponde a:

\[E_{c,\lambda} = E_{c,T} \land E_{p,\lambda} = E_{p,T}\]

Ya que la energía mecánica se conserva, es fácil darse cuenta que la energía
mecánica promedio es igual a la energía mecánica en cualquier tiempo:

\[\prom{E_m}{T} = E_m(t)\]

De igual forma, la energía mecánica en un instante de tiempo, se define
como la suma de la energía cinética y potencial en ese mismo instante,
por lo que el promedio de la energía mecánica será la suma del promedio
de la potencial y la cinética. El cálculo de todo esto resulta mucho
más fácil una vez se muestra que la energía potencial promedio
y la energía cinética promedio tienen el mismo valor en el periodo
temporal $T$ (\hyperref[EnergP]{Demostración}). Debido a esta igualdad,
se obtiene el siguiente resultado:
\[
    \begin{derivation}
            \res{E_T = 2\prom{E_c}{T} = 2\prom{E_p}{T}}\\
        \To\\
            \res{E_\lambda = 2\prom{E_c}{\lambda} = 2\prom{E_p}{\lambda}}
    \end{derivation}    
\]

De esta forma, si consideramos un tramo de longitud $\Delta x$ de la
cuerda, recordando que $\mu = \frac{\Delta m}{\Delta x}$, se podría
aproximar la energía cinética en este tramo tomando uno de los $x$
en este:
\[
    \begin{derivation}
            \res{E_{c, \Delta x_k}}\\
        \why[=]{$\gamma = \pi - kx_k^* - \phi$}\\
            \res{ \frac{1}{2}\mu\Delta x_k(A w \cos(wt + \gamma))^2 }
    \end{derivation}
\]

Ahora, consideramos la suma de la energía de $n$ tramos de cuerda entre
$x=0$ y $x=\lambda$:

\[
    \begin{derivation}
            \res{E_{c,\lambda} \approx \sum_{i=0}^{n} 
                    \frac{1}{2}\mu(A w \cos(wt + \gamma))^2\Delta x_i}\\
        \why[\To]{al aumentar la cantidad de intervalos,
                 disminuye $\Delta x$ y la suma coincide}\\
            \res{E_{c\lambda} = \int_0^\lambda \frac{1}{2}\mu 
                (Aw \cos(wt + \gamma))^2 \diff}\\
        \equiv\\
            \res{ E_{c,\lambda} = \int_0^\lambda \frac{1}{2}\mu 
                (Aw \cos(wt + \pi - kx - \phi))^2 \diff }\\
        \why{ $\cos(\alpha + \pi - \beta) = \cos(\beta - \pi - \alpha)$}\\
            \res{ E_{c,\lambda} = \frac{\mu A^2 w^2}{2}
                \int_0^\lambda \cos(kx + (\phi - wt - \pi))^2 \diff }\\
        \equiv\\
            \res{ E_{c,\lambda} = \frac{\mu A^2 w^2}{4}\lambda }
    \end{derivation}
\]

Como $E_m = 2E_{c, \lambda} = 2E_{p, \lambda}$, entonces,
$E_m = \dfrac{\mu A^2 w^2}{2}\lambda$
\marginpar{\raggedright ¿Por qué esta igualdad de energías?}