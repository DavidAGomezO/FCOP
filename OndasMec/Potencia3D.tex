\section{Potencia en Tres Dimensiones}

Como en medios tridimensionales, no existe una dirección única, hace
falta especificarla para la onda, lo cual implica también, especificar
la energía que esta transporta en dicha dirección. En este caso, se
habla de Intensidad, a diferencia de la Potencia. La intensidad se
define como la razón entre energía y, unidad de área por unidad de
tiempo.

%Dibujo
\begin{center}
    \begin{tikzpicture}
        \draw (0,0) -- (0,3) -- (5,2) -- (5,-1) -- (0,0);
        \draw (-3.5,0.2) -- (-0.5,0.9) -- (-1,0.3);
        \draw[ -> ] (-1,0.3) -- (2,1);
    \end{tikzpicture}
\end{center}

\[I_n = \frac{P}{\text{Área}} = \frac{1}{2}\rho A^2 w^2 v\]

Donde $\rho$ es la densidad volumétrica de masa del medio.

\subsection{Emisores Isotrópicos}

Un emisor isotrópico es aquel que, en cualquier dirección, emite la
misma onda. Más específicamente, si se considera una esfera de radio
$x_0$, la función de onda tiene el mismo valor en todos sus puntos

El frente de onda es la región del espacio en la que la función de onda
de un emisor tiene el mismo valor. Para un emisor isotrópico, el frente
de onda es esférico.