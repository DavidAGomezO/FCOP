\subsection{Ejemplo Numérico}

Una cuerda con densidad lineal de masa $\mu=\qty{1}{\g \per \m}$
y una tensión de magnitud $T_n = \qty{19}{\N}$ la cual tiene su
cinemática descrita por 
$y(x,t) = (\qty{3}{\cm})\sin\left(4x - 5t + \dfrac{\pi}{6}\right)$

Como la función en el caso general es
\[y(x,t) = A\sin(kx - kvt + \phi)\]
se pueden obtener inmediatamente algunas de las variables por simple
comparación. Así,

\begin{center}
    \begin{tabular}{ | >{$}l<{$} | l | }
        \hline
        A       & \qty{3}{\cm}\\
        \hline
        k       & \qty{4}{\radian \per \m}\\
        \hline
        w       & \qty{5}{\radian \per \s}\\
        \hline
        v       & $5/k =$ \qty{1.25}{\m \per \s}\\
        \hline
        \phi    & $\pi/6$\\
        \hline
    \end{tabular}
\end{center}

Para hallar la potencia, simplemente se reemplazan los datos en el
resultado obtenido anteriormente:

\[
    \begin{derivation}
            \res{P = \frac{1}{2}\mu A^2 w^2 v}\\
        \why{ Para obtener Watts, primero se debe dejar todo en unidades básicas }\\
            \res{ P = \frac{1}{2}(\qty{1e-3}{\kg\per\m})
                (\qty{3e-2}{\m})^2 (\qty{5}{\radian\per\s})^2
                (\qty{1.25}{\m\per\s}) }\\
        \equiv\\
            \res{ P = \qty{1.40625e-5}{\watt} }
    \end{derivation}
\]
