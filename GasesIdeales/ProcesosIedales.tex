\section[Algunos Procesos Ideales como referencia a Procesos Reales]{
Algunos Procesos Ideales\\
Como referencia a Procesos Reales}

Todos los procesos que se van a mostrar, serán de gases ideales, por lo
que las ecuaciones mostradas anteriormente aplican en cada uno de estos.
A su vez, se tomará que no aumenta la cantidad de partículas. Esto es,
$n$ y $N$ son constantes en cada uno de los procesos.

\begin{enumerate}
  \item Proceso Isotérmico:
        En este proceso, la temperatura permanece constante, y se le
        dará el nombre de $T_0$. Así, $P\,V = n\,R\,T_0$. Lo que indíca
        que $P\,V$ debe ser constante. En otras palabras, $P$ y $V$
        mantienen una relación inversa bajo la condición de que la
        temperatura permanezca constante.

        \begin{center}
          \begin{tikzpicture}
            \begin{axis}[
              xmin=-0.5,
              xmax=3,
              ymin=-0.5,
              ymax=3,
              xlabel = $V$,
              ylabel = $P$,
              axis lines=center,
              xtick={0},
              ytick={0},
            ]

            \addplot[
              domain=0.4:2.5,
              samples=200,
              color=azulito
            ]{
              1/x
            };
            \draw[densely dashed] (0.4,2.5) -- (0.4,-0.1)
              node[below] {$V_0$} ;
            \draw[densely dashed] (2.5,0.4) -- (2.5,-0.1)
              node[below] {$V_f$};
            \draw[densely dashed] (2.5,0.4) -- (-0.1,0.4)
              node[left] {$P_f$};
            \draw[densely dashed] (0.4,2.5) -- (-0.1,2.5)
              node[left] {$P_0$};
            \end{axis}
          \end{tikzpicture}
        \end{center}

        Como se tiene esta relación inversa de proporcionalidad,
        la ecuación que se deduce es:
        \[P_0\,V_0 = P_f\,V_F\]

        Pasando ahora a las transferencias de energía que se deben
        presentar en este tipo de procesos.
        \[
          \begin{derivation}
              \res{ \Updelta U = Q + T_r }\\
            \why{Como la energía varía con la temperatura, el cambio es nulo}\\
              \res{ Q = -T_r }\\
            \why{ Tomando el proceso de $V_0$ a $V_f$, el gas pierde energía }\\
              \res{ Q = -\Int{V_0,V_f}P\,\diff{V} }\\
            \why{ Tomando $\upalpha$ como la constante de proporcionalidad }\\
              \res{ Q = -\Int{V_0,V_f} \frac{\upalpha}{V}\,\diff{V} }\\
            \equiv\\
              \res{ Q = \upalpha\,\ln\left(\frac{V_0}{V_f}\right) }
          \end{derivation}
        \]
        Lo que dice este resultado es que, para poder mantener la
        temperatura del gas constante, se debe aplicar calor sobre este,
        y dicho calor debe ser igual al trabajo realizado por el gas
        sobre el entorno.
  \item Proceso Isomérico:
        En este proceso, el volumen permanece constante y se le dará el
        nombre de $V_0$. Así $P\,V_0 = n\,R\,T$. Lo que indica que $P$
        mantiene una proporción directa con $T$.
        
        \begin{center}
          \begin{tikzpicture}
            \begin{axis}[
              xmin=-0.5,
              xmax=1,
              ymin=-0.5,
              ymax=3,
              xlabel = $V$,
              ylabel = $P$,
              axis lines=center,
              xtick={0},
              ytick={0},
            ]
            \draw[densely dashed] (0.4,0.4) -- (0.4,-0.1)
              node[below] {$V_0$};
            \draw[densely dashed] (0.4,0.4) -- (-0.1,0.4)
              node[left] {$P_f$};
            \draw[densely dashed] (0.4,2.5) -- (-0.1,2.5)
              node[left] {$P_0$};
            \draw[azulito] (0.4,2.5) -- (0.4,0.4);
            \end{axis}
          \end{tikzpicture}
        \end{center}

        Por esta proporcionalidad directa, la ecuación que se deduce es:
        \[\frac{T_f}{T_0} = \frac{P_f}{P_0}\]

        Pasando ahora a las transferencias de energía que se deben
        presentar en este tipo de procesos.
        \[
          \begin{derivation}
              \res{ \Updelta U = Q + T_r }\\
            \why{Como el volumen no varía, el trabajo es nulo}\\
              \res{ \Updelta U = Q }\\
            \why{ $Q$ se puede expresar en función de masa y calor específico}\\
              \res{ \Updelta = n\,C_V\,(T_f - T_0) }
          \end{derivation}
        \]

        Cabe resaltar que el calor específico en los gases ideales,
        depende del proceso, por lo que se los nombrará con un
        subíndice relacionado a la variable que caracterice dicho
        proceso.
  \item Proceso Isobárico:
        En este proceso la presión se mantiene constante, y se le dará
        el nombre de $P_0$. Así $P_0\,V = n\,R\,T$. Lo que indíca que
        $V$ y $T$ mantienen una relación de proporción directa.
        \begin{center}
          \begin{tikzpicture}
            \begin{axis}[
              xmin=-0.5,
              xmax=3,
              ymin=-0.5,
              ymax=3,
              xlabel = $V$,
              ylabel = $P$,
              axis lines=center,
              xtick={0},
              ytick={0},
            ]


            \draw[densely dashed] (0.4,2.5) -- (0.4,-0.1)
              node[below] {$V_0$} ;
            \draw[densely dashed] (2.5,2.5) -- (2.5,-0.1)
              node[below] {$V_f$};
            \draw[densely dashed] (0.4,2.5) -- (-0.1,2.5)
              node[left] {$P_0$};
            \draw[azulito] (0.4,2.5) -- (2.5,2.5);
            \end{axis}
          \end{tikzpicture}
        \end{center}

        Pasando ahora a las transferencias de energía que se deben
        presentar en este tipo de procesos.
        \[
          \begin{derivation}
              \res{ \Updelta U = Q + T_r }\\
            \equiv\\
              \res{ \Updelta U = n\,C_p\,(T_f - T_0) - \Int{V_0,V_f}P\,\diff{V} }\\
            \why{ La presión es constante }\\
              \res{ \Updelta U = n\,C_p\,(T_f - T_0) - P_0\,(V_f - V_0) }
          \end{derivation}
        \]
  \item Proceso Adiabático:
        En estos procesos, no hay transferencias de energía tipo calor,
        lo que resulta en que la relación entre $P$ y $V$ sea inversa
        bajo un exponente. Es decir, existen $\upgamma$ y $\upalpha$
        para los cuales $P\,V^\upgamma = \upalpha$. Específicamente
        \[\upgamma = \frac{C_P}{C_V}\]
        y se conoce como constante adiabática.
        
        La gráfica en azul es la correspondiente a este proceso, la
        verde es la del proceso isotérmico.

        \begin{center}
          \begin{tikzpicture}
            \begin{axis}[
              xmin=-0.5,
              xmax=4,
              ymin=-0.2,
              ymax=1.2,
              xlabel = $V$,
              ylabel = $P$,
              axis lines=center,
              xtick={0},
              ytick={0},
            ]

            \addplot[
              domain=1:3.1,
              samples=200,
              opacity = 0.5,
              color=verdecito
            ]{
              1/x
            };
            \addplot[
              domain=1:3.1,
              samples=300,
              azulito
            ]{
              1/(x^2)
            };
            \draw[densely dashed] (1,1) -- (1,-0.1)
              node[below] {$V_0$} ;
            \draw[densely dashed] (3.1,{1/3.1^2}) -- (3.1,-0.1)
              node[below] {$V_f$};
            \draw[densely dashed] (3.1,{1/3.1^2}) -- (-0.1,{1/3.1^2})
              node[left] {$P_f$};
            \draw[densely dashed] (1,1) -- (-0.1,1)
              node[left] {$P_0$};
            \end{axis}
          \end{tikzpicture}
        \end{center}

        Bajo esta relación entre $P$ y $V$, se deduce la siguiente ecuación:

        \[\frac{P_0}{P_f} = \left(\frac{V_f}{V_0}\right)^\upgamma\]

        Pasando ahora a las transferencias de energía que se deben
        presentar en este tipo de procesos.

        \[
          \begin{derivation}
              \res{ \Updelta U = Q + T_r }\\
            \why{ $Q=0$ }\\
              \res{ \Updelta U = T_r }\\
            \equiv\\
              \res{ \Updelta U = \Int{V_0,V_f}P\,\diff{V} }\\
            \equiv\\
              \res{ \Updelta U = \Int{V_0,V_f}\frac{\upalpha}{V^\upgamma}\,\diff{V} }\\
            \why{$\upgamma \not= 1$}\\
              \res{ \Updelta U = 
                \frac{\upalpha}{1 - \upgamma}\left(V_f^{1-\upgamma} - V_0^{1-\upgamma}\right) }
          \end{derivation}
        \]
\end{enumerate}