\section{Máquina Térmica}

Una máquina térmica es un sistema el cual recibe energía de tipo
calor desde un reservorio a una temperatura $T_A$, generar con esta
trabajo mecánico útil, y entrega energía de tipo calor a otro reservorio
a temperatura $T_B$. Un reservorio es un sistema que permite el paso de
energía de tipo calor sin alterar su temperatura.

\begin{center}
  \begin{tikzpicture}
    \draw[domain=0:360, samples=500] plot ({3*cos(\x)},{sin(\x)+4});
      \node at (0,4) {$T_A$};
    \draw[domain=0:360, samples=500] plot ({cos(\x)},{sin(\x)});
      \node at (0,0) {Máquina};
    \draw[domain=0:360, samples=500] plot ({3*cos(\x)},{sin(\x)-4});
      \node at (0,-4) {$T_B$};

    \draw[-stealth] ({3*cos(deg(-0.5*pi))},{sin(deg(-0.5*pi))+4}) 
    -- (-0.2, 1.8) -- (0.2,2.2) -- ({cos(deg(0.5*pi))},{sin(deg(0.5*pi))});
      \node at (0.6,2) {$Q_A$};
    \draw[-stealth]  ({cos(deg(-0.5*pi))},{sin(deg(-0.5*pi))})
    -- (-0.2, -2.2) -- (0.2,-1.8) -- ({3*cos(deg(0.5*pi))},{sin(deg(0.5*pi))-4});
      \node at (0.6,-2) {$Q_B$};
    \draw[-stealth] (1,0) -- (2,0) node[right] {$T_{r,u}$ Trabajo mecánico útil};
  \end{tikzpicture}
\end{center}

Toda máquina térmica requiere de sustancia de trabajo, esta sustancia absorbe la energía
$Q_A$ y es la que, como intermediaria, permite la generación del trabajo útil. La
eficiencia de una máquina térmica se puede medir en razón de cuanta energía de la que le
entra es producida como trabajo, esto es:
\[\text{Eficiencia:} e = \frac{T_{r,u}}{Q_A}\]
Los intercambios de energía se dan en la sustancia de trabajo, la cual siempre está operando
en un ciclo. Esto es, en un plano presión - volúmen, se pueden trazar diferentes procesos
del funcionamiento de la máquina, y la sustancia de trabajo va a cambiar entre una seria de
estados, pero siempre en un ciclo. Es decir, el último estado finaliza en donde empezó el primero.

Ya que la sustancia de trabajo tiene esta propiedad. Considerando todo un ciclo,

\begin{longderivation}
    \res{ \Updelta U = Q + T_r }\\
  \why{ Se finaliza en el mismo estado en el que se inició }\\
    \res{ Q = -T_r }
\end{longderivation}

Como $Q = Q_A - Q_B$ y $T_{r,u} = -T_r$ entonces la eficiencia se puede escribir
como:
\[e = 1 - \frac{Q_B}{Q_A}\]

Un ejemplo de una máquina térmica es un motor de combustión, donde la sustancia de trabajo
viene a ser la mezcla del combustible y el aire.