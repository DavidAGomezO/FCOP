\section{El Calor Específico Depende del Proceso y el Valor de $\upgamma$}

Considerando un mismo gas, y dos procesos diferentes para el mismo, uno
isobárico y otro isomérico. Tomando que los cambios de temperatura sean
los mismos ($T_1$ y $T_0$), que el proceso isométrico mantiene un
volumen $V_0$, con presiones $P_0$ a $P_f$ y el proceso isométrico
se mantiene en la presión $P_f$ con volumenes $V_f$ a $V_0$. Entonces
se tiene que:

\begin{center}
  \begin{tikzpicture}
    \begin{axis}[
      xmin=-0.5,
      xmax=3,
      ymin=-0.5,
      ymax=3,
      xlabel = $V$,
      ylabel = $P$,
      axis lines=center,
      xtick={0},
      ytick={0},
    ]

    \addplot[
      domain=0.095:0.5,
      samples=600
    ]{
      0.25/x
    };

    \addplot[
      domain=0.4:2.5,
      samples=300
    ]{
      1.25/x
    };

    \draw[azulito] (0.5,0.5) -- (0.5,2.5);
    \draw[verdecito] (0.5,0.5) -- (2.5, 0.5);

    \draw[densely dashed] (0.5,0.5) -- (-0.1,0.5)
      node[left] {$P_f$};
    \draw[densely dashed] (0.5,2.5) -- (-0.1,2.5)
      node[left] {$P_0$};
    \draw[densely dashed] (0.5,0.5) -- (0.5,-0.1)
      node[below] {$V_0$};
    \draw[densely dashed] (2.5,0.5) -- (2.5,-0.1)
      node[below] {$V_f$};
    \end{axis}
  \end{tikzpicture}
\end{center}

\begin{enumerate}
  \item En el proceso isométrico
        \[\Updelta U_V = n\,C_V\,(T_1 - T_0)\]
  \item En el proceso isobárico
        \[\Updelta U_P = n\,C_P\,(T_1 - T_0) - P_f(V_0 - V_f)\]
        \[P_f\,(V_0 - V_1) = n\,R\,(T_1 - T_0)\]
\end{enumerate}

Como la energía está en función de la temperatura, ambos procesos
tienen el mismo cambio de temperatura y son el mismo gas, se tiene que:
\[
  \begin{derivation}
      \res{ \Updelta U_V = \Updelta U_P }\\
    \equiv\\
      \res{ n\,C_V\,(T_1 - T_0) = n\,C_P\,(T_1 - T_0) - P_f(V_0 - V_f) }\\
    \equiv\\
      \res{ n\,C_V\,(T_1 - T_0) = n\,C_P\,(T_1 - T_0) - n\,R\,(T_1 - T_0) }\\
    \equiv\\
      \res{ C_P - C_V = R }
  \end{derivation}
\]