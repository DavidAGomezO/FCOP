\section{El Calor Específico Depende del Proceso y el Valor de $\upgamma$}

Considerando un mismo gas, y dos procesos diferentes para el mismo, uno
isobárico y otro isomérico. Tomando que los cambios de temperatura sean
los mismos ($T_1$ y $T_0$), que el proceso isométrico mantiene un
volumen $V_0$, con presiones $P_0$ a $P_f$ y el proceso isométrico
se mantiene en la presión $P_f$ con volumenes $V_f$ a $V_0$. Entonces
se tiene que:

\begin{center}
  \begin{tikzpicture}
    \begin{axis}[
      xmin=-0.5,
      xmax=3,
      ymin=-0.5,
      ymax=3,
      xlabel = $V$,
      ylabel = $P$,
      axis lines=center,
      xtick={0},
      ytick={0},
    ]

    \addplot[
      domain=0.095:0.5,
      samples=600
    ]{
      0.25/x
    };

    \addplot[
      domain=0.4:2.5,
      samples=300
    ]{
      1.25/x
    };

    \draw[azulito] (0.5,0.5) -- (0.5,2.5);
    \draw[verdecito] (0.5,0.5) -- (2.5, 0.5);

    \draw[densely dashed] (0.5,0.5) -- (-0.1,0.5)
      node[left] {$P_f$};
    \draw[densely dashed] (0.5,2.5) -- (-0.1,2.5)
      node[left] {$P_0$};
    \draw[densely dashed] (0.5,0.5) -- (0.5,-0.1)
      node[below] {$V_0$};
    \draw[densely dashed] (2.5,0.5) -- (2.5,-0.1)
      node[below] {$V_f$};
    \end{axis}
  \end{tikzpicture}
\end{center}

\begin{enumerate}
  \item En el proceso isométrico
        \[\Updelta U_V = n\,C_V\,(T_1 - T_0)\]
  \item En el proceso isobárico
        \[\Updelta U_P = n\,C_P\,(T_1 - T_0) - P_f(V_0 - V_f)\]
        \[P_f\,(V_0 - V_1) = n\,R\,(T_1 - T_0)\]
\end{enumerate}

Como la energía está en función de la temperatura, ambos procesos
tienen el mismo cambio de temperatura y son el mismo gas, se tiene que:
\[
  \begin{derivation}
      \res{ \Updelta U_V = \Updelta U_P }\\
    \equiv\\
      \res{ n\,C_V\,(T_1 - T_0) = n\,C_P\,(T_1 - T_0) - P_f(V_0 - V_f) }\\
    \equiv\\
      \res{ n\,C_V\,(T_1 - T_0) = n\,C_P\,(T_1 - T_0) - n\,R\,(T_1 - T_0) }\\
    \equiv\\
      \res{ C_P - C_V = R }
  \end{derivation}
\]

Considerando un mismo gas, dos procesos diferentes para el mismo, uno
adiabático y otro isomérico. Tomando que los ccambios de la temeperatura sean
los miamos ($T_1$ y $T_2$), que el proceso isomérico mantiene volumen $V_f$.

\begin{center}
  \begin{tikzpicture}
    \begin{axis}[
      xmin=-1,
      xmax=11,
      ymin=-0.25,
      ymax=1.2,
      xlabel = $V$,
      ylabel = $P$,
      axis lines=center,
      xtick={0},
      ytick={0},
    ]

    \addplot[
      domain=1:9.61,
      samples=200,
      opacity = 0.5,
    ]{
      1/x
    };
    \addplot[
      domain=1:3.1,
      samples=300,
      azulito
    ]{
      1/(x^2)
    };
    \draw[densely dashed] (3.1,{1/3.1^2}) -- (3.1,-0.1)
      node[below] {$V_f$};
    \draw[densely dashed] (9.61, {1/3^2}) -- (9.61,-0.1)
      node[below] {$V_f'$};
    \draw[densely dashed] (9.61,{1/3.1^2}) -- (-0.1,{1/3.1^2})
      node[left] {$P_f$};
    \draw[densely dashed] (1,1) -- (-0.1,1)
      node[left] {$P_0$};
    \draw[verdecito] (3.1, {1/3.1^2}) -- (3.1, {1/3.1});
    \end{axis}
  \end{tikzpicture}
\end{center}

Se tiene entonces que:
\[
  \begin{derivation}
      \res{
        \begin{siseq}
          \Updelta U_V  &= Q_V        &&= n\,C_V\,\Updelta T\\
          \diff{U_Q}    &= \diff{T_r} &&= -P\diff{V}
        \end{siseq}
      }\\
    \To\\
      \res{
        \begin{siseq}
          Q_V         &= n\,C_V\,\Updelta T\\
          -P\diff{V}  &= \diff{U_V}
        \end{siseq}
      }\\
    \To\\
      \res{
        \begin{siseq}
          \diff{Q_V}  &= n\,C_V\,\diff{T}\\
          -P\diff{V}  &= \diff{U_V}
        \end{siseq}
      }
  \end{derivation}
\]

Por otro lado
\begin{longderivation}
    \res{ \diff{(P\,V)} = n\,R\,\diff{T}}\\
  \equiv\\
    \res{ \diff{P}\,V + P\,\diff{V} = n\,R\,\diff{T} }\\
  \equiv\\
    \res{ \diff{P}\,V + P\,\diff{V} = -n\,R\frac{P\,\diff{V}}{n\,C_V} }\\
  \why{ $R = C_P - C_V$ }\\
    \res{ \diff{P}\,V + P\,\diff{V} = - (C_P - C_V)\frac{P\,\diff{V}}{C_V} }\\
  \why{ $\upgamma = C_P/C_V$ }\\
    \res{ \diff{P}\,V + P\,\diff{V} = (1 - \upgamma)\,P\,\diff{V} }\\
  \equiv\\
    \res{ \frac{\diff{P}\,V + P\,\diff{V}}{PV} = (1 - \upgamma)\frac{P\,\diff{V}}{PV} }\\
  \equiv\\
    \res{ \frac{\diff{P}}{P} + \frac{\diff{V}}{V} = (1 - \upgamma)\frac{\diff{V}}{V} }\\
  \equiv\\
    \res{ \frac{\diff{P}}{P} = -\upgamma\frac{\diff{V}}{V} }\\
  \To\\
    \res{ \Int{P_0,P_f}\frac{1}{P}\diff{P} = -\upgamma\Int{V_0,V_f}\frac{1}{V}\diff{V} }\\
  \equiv\\
    \res{ \ln\left(\frac{P_f}{P_0}\right) = -\upgamma\ln\left(\frac{V_f}{V_0}\right) }\\
  \equiv\\
    \res{ \frac{P_0}{P_f} = \left(\frac{V_f}{V_0}\right)^\upgamma }
\end{longderivation}
