\section{Gases Ideales}

Los gases ideales son un sistema termodinámico que se compone de varias
partículas. Se caracteriza por una temperatura ($T$) un volumen ($V$),
una cantidad de masa, medida en moles $n$ y una presión ejercida al
entorno.

\begin{center}
  \begin{tikzpicture}
    % ~~~~ Figuras ~~~~ %
    \draw[domain=0:360, samples=500] plot ({6*cos(\x)},{4.5*sin(\x)});
    \draw[amarillito] (0,0) circle (3);

    % ~~~~ Variables ~~~~ %
    \node at (0,2) {Temperatura $T$};
    \node at (0,-1) {Volumen $V$};
    \node at (0,0) {\textcolor{amarillito}{Gas Ideal}};
    \node at (0,-2) {Moles $n$};

    % ~~~~ Flechas ~~~~ %
    \draw[-stealth, amarillito] ({3*cos(deg(-0.25*pi))},{3*sin(deg(-0.25*pi))})
      -- (3,{tan(deg(-0.25*pi))*(3-3*cos(deg(-0.25*pi))) + 3*sin(deg(-0.25*pi))})
      node[midway, xshift=32] {Presión, $P$};
  \end{tikzpicture}
\end{center}

La ecuación de gases ideales, obtenida de forma experimental, relaciona
estas variables de la siguiente manera:
\[P\,V = n\,R\,T\]
Donde $R$ es una constante que depende de las unidades empleadas en las
demás variables. Su valor bajo las unidades del sistema internacional es:
\[R = \qty{8.314}{\joule\per{\mol \kelvin}}\]

\subsection{Características de los Gases Ideales}
\begin{enumerate}
  \item Las fuerzas entre sus partes (átomos, moléculas) son nulas.
  \item La única interacción entre sus partes se da cuando colisionan.
  \item Las colisiones entre partículas no afectan la energía potencial
        dentro de estas.
  \item Los intercambios de energía con el entorno solo se dan en forma
        de energía cinética de las partículas del gas, por lo que estos
        intercambios solo afectan su temperatura. De esto se deduce que
        la energía de un gas ideal se puede expresar como función de su
        temperatura.
\end{enumerate}

Con el desarrollo de la física estadística, se encontró otra expresión
para estos gases, donde se toma ahora el número de partículas ($N$) que
componen al gas, en lugar del número de moles; y la constante de
Boltzmann ($k_B = \qty{1.38e-23}{\joule\per\kelvin}$), en lugar de la
constante $R$:
\[P\,V = N\,k_B\,T\]

Igualando ambas ecuaciones se puede concluir que $N\,k_B = n\,R$. De
este resultado se puede obtener la constante de Avogadro fácilmente al
tomar $n=1$.