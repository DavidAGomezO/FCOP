\section{Manifestaciones de la energía en sistemas termodinámicos}

La energía se manifiesta de tres maneras en un sistema:
\begin{enumerate}
    \item \textcolor{amarillito}{Energía cinética de sus partes}
    \item \textcolor{azulito}{Energía potencial entre sus partes}
    \item Energía potencial dentro de sus partes
\end{enumerate}

\begin{center}
    \begin{tikzpicture}
        % ~~~~ Partes ~~~~ %
        \draw (0,0) circle (4);
        \draw (-2.5,1) circle (0.5);
        \draw (-1, 2.8) circle (0.5);
        \draw (1.5,2.5) circle (0.5);
        \draw (1,-1.5) circle (1.75);
        \draw (0.3, -2.5) circle (0.3);
        \draw (2, -0.8) circle (0.3);

        % ~~~~ Flechas ~~~~ %
        \draw[ ->,amarillito ] (-2.85355339,0.6464466) -- (-3.5,0)
            node [midway, yshift=-4,xshift=8] {$v_1$};
        \draw[ ->, amarillito ] (-1.25,2.366987) -- (-1.75,1.50096189)
            node [midway, yshift=0,xshift=8] {$v_2$};
        \draw[ ->, amarillito ] (1.85355339,2.14655661) -- (3,1)
            node [midway, yshift=-2,xshift=-8] {$v_3$};

            %TODO dibujar el resto de flechas
    \end{tikzpicture}
\end{center}

Entre dos sistemas termodinámicos, a diferente temperatura, los cuales
se encuentran en contacto térmico; La energía de tipo calor ($Q$) que
cede el sistema a mayor temperatura causa una disminución de la energía
cinética de sus partes, mientras que el sistema receptor la absorbe a
modo de energía cinética o potencial (cualquiera de las dos potenciales).

\begin{center}
    \begin{tikzpicture}
        % ~~~~ Sistemas ~~~~ %
        \draw (0,0) circle (1)
            node {$A\,,\,T_A$};
        \draw (4,0) circle (1)
            node {$B\,,\,T_B$};
        % ~~~~ Rayos/Rep.Energia ~~~~ %
        \draw[amarillito, ->] (1, 0.5) -- (2, 0.6) -- (1.8, 0.4) -- (3, 0.5)
            node[midway,xshift=-12.5, yshift=10] {Energía};
        \draw[amarillito, ->] (2.9,-0.1) -- (1.8,-0.2) -- (2,0) -- (1.1,-0.1);
    \end{tikzpicture}
\end{center}

Si la temperatura del sistema receptor no es cercana a las temperaturas
de transición de fase de este, la energía que que absorbe lo hace en
modo de energía cinética, lo que significa que su temperatura cambia de
forma lineal.

Sean $T$ la temperatura del sistema receptor como función del tiempo,
$T_0$ la temperatura inicial de este, $m$ su masa y $c$ el calor
específico de la sustancia. Entonces:
\[
    \begin{derivation}
            \res{ Q = m\,c\,(T-T_0)}\\
        \equiv\\
            \res{ \odv{Q}{t} = m\,c }
    \end{derivation}
\]

El calor específio, es una constante que indica la cantidad de energía
requerida para hacer que el sistema cambie su temperatura.
Nótese que $[c] = \si{\joule\per{\kelvin\kg}}$

Por otro lado, si la temperatura del sistema receptor, es cercana o
igual a las temperaturas de transición de fase de este, la energía que
absorbe lo hace en modo de energía potencial, lo que significa que su
temperatura no varía, manteniendo una temperatura constante durante
dicha transición de fase.

Sea $m$ la masa del sitema receptor y $L_f$ el calor latente de
transición de fase. Entonces:
\[Q = m\,L_f\]
El calor latente mide la cantidad de enería requerida para que una
cantidad de masa experimente la transición.
Nótese que $[L_f] = \si{\joule\per\kg}$