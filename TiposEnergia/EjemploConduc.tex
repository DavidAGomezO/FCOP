\subsection{Ejemplo Numéríco: Conductividad}

Una ventana de espesor $L=\qty{2}{\mm}$, con sección transversal
$A_C = \qty{2e6}{\mm^2}$, conductividad térmica
$k = \qty{0.8}{\watt\per{\m\kelvin}}$. Sometida a una temperatura
constante por un lado $T_A = \qty{20}{\degreeCelsius}$ y por el otro a
una temperatura $T_B=\qty{15}{\degreeCelsius}$. Hallar $T$ en
$x=\qty{1}{\mm}$. Bajo estas condiciones, se tiene que el término
$\ds\odv{T}{x}$ es constante, pues de no ser así, se podría pensar en cada
sección de la ventana como un subsistema, el cual estaría cambiando de
temperatura, lo que, por segunda ley de la termidinámica, entraría en
contradicción con la suposición de que $T_A$ y $T_B$ permanecen
constantes.

Así, se tiene entonces que la temperatura en función de la distancia se
comporta de manera lineal. Como se conocen las temperaturas de los
extemos en contacto con ambas temperaturas, se pueden usar para
determinar este comportamiento:
\[
    \begin{derivation}
            \res{ \odv{T}{x} = \alpha }\\
        \equiv\\
            \res{ \odv{T}{x} = \frac{\Delta T}{\Delta x} }\\
        \To\\
            \res{ \odv{T}{x} = \frac{T_A - T_B}{-L} }\\
        \equiv\\
            \res{ \odv{T}{x} = -\qty{2.5}{\degreeCelsius\per\mm} }
    \end{derivation}
\]

Por la misma razón, se debe mantener la igualdad cuando se toma, en vez
de $(L,T_B)$ para hallar la razón, $(x,T(x))$, en especial, para el $x$
de interés:
\[
    \begin{derivation}
            \res{ -\qty{2.5}{\degreeCelsius\per\mm} 
            = \frac{\qty{20}{\degreeCelsius} - T(\qty{2}{\mm})}{-\qty{2}{\mm}} }\\
        \equiv\\
            \res{ T(\qty{2}{\mm}) = \qty{15}{\degreeCelsius} }
    \end{derivation}
\]

Por otra parte, como se tienen los valores de $k$, $A_C$ y 
$\ds\odv{T}{x}$, entonces
\[
    \begin{derivation}
            \res{ \odv{Q}{t} = k\,A_C\,\odv{T}{x} }\\
        \equiv\\
            \res{ \odv{Q}{t} = -\qty{4e3}{\watt} }
    \end{derivation}
\]

Volviendo a un caso general, la ecuación bajo la condición de que las
temperaturas de los extremos permanece constante se puede escribir como
\[\odv{Q}{t} = A_C\,\frac{T_A - T_B}{L/k}\]
Donde $\dfrac{L}{k}$ se conoce como la resistencia térmica ($R$). Esta
resistencia, al depender de $k$ y $L$, caracterizan una lámina de un
material. En un sistema donde hay múltiples lámines en serie, la
resistencia térmica total es la suma de las resistencias de sus láminas.