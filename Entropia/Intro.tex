\section{Leyes de la Termodinámica}

Es importante para la definición de la entropía, tener presentes estas
leyes, por lo que se nombrarán nuevamente aquí.

\begin{enumerate}
  \item $\Updelta U = \Updelta Q + \Updelta T_r$
        El cambio de la energía interna $U$ de un sistema debe ser
        igual a la energía de tipo calor $Q$ transferida o recibida por
        el sistema sumada al trabajo realizado sobre o hecho por el
        sistema. Recordando que la convención de los signos es:
        \begin{itemize}
          \item $Q$ es positivo cuando es transferido al sistema, y es
                negativo cuando transferido por el sistema.
          \item $T_r$ es positivo cuando se realiza sobre el sistema, y
                es negativo cuando es realizado por el sistema.
        \end{itemize}
  \item Un proceso cuyo resultado neto sea de tomar energía de tipo
        calor desde un reservorio y transformarla en trabajo es
        imposible.
\end{enumerate}

Pasando ahora a la definición de entropía:\\
Si una cantidad de calor $\Updelta Q$ es añadida reverisblemente a un
sistema a temperatura $T$, el cambio en su entropía $S$ es:

\[\Updelta S = \frac{\Updelta Q}{T}\]

La tercera ley de la termodinámica, que es también otra definción de la
entropía, se escribe entonces como:

\begin{enumerate}
  \setcounter{enumi}{2}
  \item Si la temperatura $T$ de un sistema tiende a $\qty{0}{\kelvin}$,
        entones $S \to 0$
\end{enumerate}

