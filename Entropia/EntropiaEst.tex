\section{La Entropía como Medida del Desorden}

El ejemplo típico para explicar la entropía en su interpretación
estadística, es considerar un gas en una especie de caja, limitado a una=
parte de esta por una barrera. Al retirar la barrera, el gas tiende
a ocupar toda la caja, esto por simple probabilidad de la posición de todas
las partículas. Para analizar esta situación con lo visto anteriormente,
nos damos cuenta que el proceso descrito es isotérmico y adiabático. Es un
proceso irreversible. Se puede, sin embargo, considerar un proceso en el que la
barrera no se retira, sino que se desliza, hasta permitir que el gas ocupe todo
el volumen de la caja, siendo este un proceso únicamente isotérmico, pasando
de un volumen $V_i$ a un volumen $V_f$.

Así, el cambio de la entropía será $\Updelta S = n\,R\,\ln\left(\frac{V_f}{V_i}\right)$

Ahora, si consideramos el volumen de cada partícula como $V_m$, para úna partícula,
las configuraciones posibles en un volumen $V_i$ serán $\dfrac{V_i}{V_m}$.
Para $N$ partículas, las configuraciones posibles son $\ds\left(\frac{V_i}{V_m}\right)^N$.
Este valor lo llamaremos $w_i$. Resulta que, en el estado inicial, la entropía
es $S_i = k_B\,\ln(w_i)$, recordando que $k_B$ es la constante de Boltzman.

Para mostrar que efectivamente esta expresión representa la entropía, se va a considerar
lo obtenido para el cambio de la entropía.

\begin{longderivation}
    \res{ \Updelta S }\\
  =\\
    \res{ n\,R\,\ln\left(\frac{V_f}{V_i}\right) }\\
  =\\
    \res{ N\,k_B\,\ln\left(\frac{V_f}{V_i}\right) }\\
  =\\
    \res{ k_B\,\ln\left(\frac{V_f}{V_i}\right)^N }\\
  =\\
    \res{ k_B\,\ln\left(\frac{w_f}{w_i}\right) }\\
  =\\
    \res{ k_B\ln(w_f) - k_B\ln(w_i) }
\end{longderivation}