\section{Características de la Entropía}

Una de las características de la entropía, es que, en un cambio
reverisble, La entropía del sistema no cambia de su estado
inicial. Por otro lado, en un proceso irreversible, el cambio de la
entropía debe ser positivo ($\Updelta S > 0$). La entropía es una
función del estado del estado termodinámico del sistema, y no depende
de cómo este llegó a dicho estado. No existe una expresión para
calcular el cambio de entropía en procesos irreversibles, sin embargo,
si se tienen procesos reversibles entre los estados del proceso
irreversible, el cambio de entropía debe coincidir.

\subsection{Cálculo de la Entropía en Procesos Reversibles}

Se va a considerar un proceso reversible entre dos estados: uno
caracterizado por presión, volumen y temperatura $P_i$, $V_i$, $T_i$.
El otro, de igual manera, por $P_f$, $V_f$, $T_f$.
\begin{center}
  \begin{tikzpicture}
    \begin{axis}[
      axis lines=center,
      xmin=-0.5,
      xmax=3,
      ymin=-1,
      ymax=7,
      xlabel=$V$,
      ylabel=$P$,
      xtick=\empty,
      ytick=\empty,
    ]
    % ~~~~ Curvas isotérmicas ~~~~~%
    \addplot[
      domain=0.8:2.5,
      samples=300,
      color=azulito,
    ]{
      6/x
    } node[pos=0.7,yshift=12] {$T_i$};

    \addplot[
      domain=0.8:2.5,
      samples=300,
      color=azulito,
    ]{
      1/x
    } node[pos=-0.15,yshift=-4] {$T_f$};
    % ~~~~ Procesos isométricos ~~~~ %
    \draw[
      postaction = {decorate},
      decoration = {
                    markings
                    , mark = at position 0.5 with {\arrow{Stealth[length = 1.5mm, bend]}}
                }
    ] (1,{6/1}) -- (1,{1/1});
    % ~~~~ Valores en ejes ~~~~ %
    \draw (0,{2/2^2}) -- (-0.2,{2/2^2}) node[left] {$P_f$};
    \draw (0,{6/1^2}) -- (-0.2,{6/1^2}) node[left] {$P_i$};
    \draw (1,0) -- (1,-0.2) node[below] {$V_i$};
    \draw (2,0) -- (2,-0.2) node[below] {$V_f$};

    % ~~~~ Lineas de los valores ~~~~ %
    \draw[dotted] (1,0) -- (1,1);
    \draw[dotted] (2,0) -- (2,{1/2});
    \draw[dotted] (2,{1/2}) -- (0,{1/2});
    \draw[dotted] (2,{1/2}) -- (0,{1/2});
    \draw[dotted] (1,{6/1}) -- (0,{6/1});
    \end{axis}
  \end{tikzpicture}
\end{center}

Para lograr calcular el cambio en la entropía, consideraremos las dos
curvas isotérmicas con temperaturas $T_i$ y $T_f$ respectivamente.
Con estas curvas, se puede apreciar mejor que el proceso isométrico
mostrado, varía entre las temperaturas $T_i$ y $T_f$, igual que el
proceso reversible. Debido a esto, los cambios de energía de ambos deben
ser iguales: $\Updelta U = \Updelta U_V$. De igual forma tenemos que
$\Updelta U_V = n\,C_V\,\Updelta T$.

Por otro lado, para nuestro proceso reversible, como pasa de un menor
volumen a un mayor volumen, el trabajo es negativo.

\[\diff{T_r} = -P\,\diff{V} = -\frac{n\,R\,T}{V}\,\diff{V}\]

Por primera ley, se tiene que:

\[
  \begin{derivation}
      \res{ \Updelta Q = \Updelta U - \Updelta T_r }\\
    \To\\
      \res{ \diff{Q} = n\,C_V\,\diff{T} + \frac{n\,R\,T}{V}\,\diff{V}}\\
    \equiv\\
      \res{ \frac{\diff{Q}}{T} = \frac{n\,C_V}{T}\,\diff{T} + \frac{n\,R}{V}\,\diff{V} }\\
    \To\\
      \res{ \Updelta S = n\,C_V\,\ln\left(\frac{T_f}{T_i}\right) + n\,R\,\ln\left(\frac{V_f}{V_i}\right)}
  \end{derivation}
\]

Esta expresión aplica para cualquier proceso reversible en gases ideales. En
los procesos adiabáticos, por ejemplo, no hace falta recurrir a esta expresión,
pues no hay transferencias de calor ($\Updelta Q = 0$), lo que significa que
el cambio en la entropía es nulo.

\subsection{Cambio de la Entropía en Procesos Irreversibles}

Se va a considerar un procesos irreversible caracterizado por las
mismas variables de antes, pasando de $i$ a $f$.

\begin{center}
  \begin{tikzpicture}
    \begin{axis}[
      axis lines=center,
      xmin=-0.5,
      xmax=3,
      ymin=-1,
      ymax=7,
      xlabel=$V$,
      ylabel=$P$,
      xtick=\empty,
      ytick=\empty,
    ]
    % ~~~~ Procesos isométricos ~~~~ %
    \draw[
      postaction = {decorate},
      decoration = {
                    markings
                    , mark = at position 0.5 with {\arrow{Stealth[length = 1.5mm, bend]}}
                },
      color=naranjita
    ] (2,{6/1}) -- (2,{1/2})
      node[midway, right] {2};
    % ~~~~ Procesos isobáricos ~~~~ %
    \draw[
      postaction = {decorate},
      decoration = {
        markings,
        mark=at position 0.5 with {\arrow{Stealth[length=1.5mm, bend]}}
      },
      color=azulito
    ] (1,6) -- (2,6)
      node[midway, above] {1};
    % ~~~~ Valores en ejes ~~~~ %
    \draw (0,{2/2^2}) -- (-0.2,{2/2^2}) node[left] {$P_f$};
    \draw (0,{6/1^2}) -- (-0.2,{6/1^2}) node[left] {$P_i$};
    \draw (1,0) -- (1,-0.2) node[below] {$V_i$};
    \draw (2,0) -- (2,-0.2) node[below] {$V_f$};

    % ~~~~ Lineas de los valores ~~~~ %
    \draw[dotted] (1,0) -- (1,6);
    \draw[dotted] (2,0) -- (2,{1/2});
    \draw[dotted] (2,{1/2}) -- (0,{1/2});
    \draw[dotted] (2,{1/2}) -- (0,{1/2});
    \draw[dotted] (1,{6/1}) -- (0,{6/1});

    \end{axis}
  \end{tikzpicture}
\end{center}

Suponiendo que se tienen los dos procesos reversibles, que juntos
serían un proceso reverisble entre los dos estados. Entonces,
se puede calcular el cambio de la entropía.\\
En el proceso isobárico, se tiene que $\diff{Q_P} = n\,C_P\,\diff{T}$\\
En el proceso isométrico, se tiene que $\diff{Q_V} = n\,C_V\,\diff{V}$

Así,
\[
  \begin{derivation}
      \res{ \Updelta S }\\
    \why[=]{ $b$ el punto donde se encuentran los dos procesos
            reversibles}\\
      \res{ \Int{i,b} \frac{1}{T}\,\diff{Q_P} + \Int{b,T_f} \frac{1}{T}\,\diff{Q_V} }\\
    =\\
      \res{ n\,C_P \Int{T_i,T_b}\frac{1}{T}\,\diff{T} +
            n\,C_V\Int{T_b,T_f}\frac{1}{T}\,\diff{T} }\\
    =\\
      \res{ n\,C_P\,\ln\left(\frac{T_b}{T_i}\right) +
            n\,C_V\,\ln\left(\frac{T_f}{T_b}\right) }
  \end{derivation}
\]

Por ejemplo, tomando un gas monoatómico con $n=2000$, $P_i=\qty{40000}{\Pa}$,
$V_i = \qty{60}{\m^3}$, $P_f = \qty{20000}{\Pa}$ y $V_f = \qty{100}{\m^3}$.

\begin{itemize}
  \item Por ley de gases ideales, se tiene que $\ds T_i = \frac{P_i\,V_i}{n\,R} 
        = \qty{144.33}{\kelvin}$
  
  \item Como se está considerando un proceso isobárico, $\ds T_b = T_i\frac{V_f}{V_i} =
        \qty{240.56}{\kelvin}$
  
  \item Por último, en el proceso isométrico, $\ds T_f = T_b\frac{P_f}{P_i} =
        \qty{120.28}{\kelvin}$
  
  \item Ya con todos los datos, reemplazando en la ecuación obtenida:
  
        \[\ds\Updelta S = 2000\,\frac{5}{2}R\,\ln\left(\frac{240.56}{144.33}\right)
        + 2000\,\frac{3}{2}R\,\ln\left(\frac{120.28}{240.56}\right)
        = \qty{4322.57}{\J\per\K}\]
\end{itemize}