\section{Máquina de Carnot}

La máquina de Carnot es una máquina térmica la cual tiene la mayor
eficiencia posible. Se caracteriza por el ciclo de su sustancia de
trabajo, la cual consta de dos procesos isotérmicos y dos adiabáticos.

\begin{center}
  \begin{tikzpicture}
    \begin{axis}[
      axis lines=center,
      xmin=-1,
      xmax=3,
      ymin=-1,
      ymax=5,
      xlabel=$V$,
      ylabel=$P$,
      xtick=\empty,
      ytick=\empty,
    ]

    \addplot[
      domain=1.5:2.5,
      samples=200
    ]{
      2/x
    };

    \addplot[
      domain=1:2,
      samples=500
    ]{
      3.6/x
    };

    \addplot[
      domain=1:1.5,
      samples=200,
      color=naranjita
    ]{
      3.6/x^2.44966029
    };

    \addplot[
      domain=1.994:2.5,
      samples=500,
      color=naranjita
    ]{
      22.23486517/x^3.63411899
    };

    \node at (1,4) {1};
    \node at (2.1,2.2) {2};
    \node at (2.6,0.8) {3};
    \node at (1.4,1.2) {4};
    \end{axis}
  \end{tikzpicture}
\end{center}

Las temperaturas correspondientes a los procesos isotérmicos
(1 a 2 y 3 a 4), serán llamadas $T_1$ y $T_3$ respectivamente.

Como los otros dos procesos son adiabáticos, por definición, la energía
de tipo calor en estos es nula. Así, la eficiencia de esta máquina es:

\[e_c = 1 - \frac{Q_B}{Q_A} = 1 - \frac{|Q_{3,4}|}{|Q_{1,2}|}\]

De los procesos isotérmicos, se tiene entonces que:

\begin{itemize}
  \item \(\ds Q_{1,2} = n\,R\,T_1\,\ln\left(\frac{V_2}{V_1}\right)\)
  \item \(\ds Q_{3,4} = n\,R\,T_3\,\ln\left(\frac{V_3}{V_4}\right)\)
\end{itemize}

De los procesos adiabáticos, se tiene que:
\begin{itemize}
  \item \(\ds T_3\,V_3^{\upgamma-1} = T_1\,V_2^{\upgamma-1}\)
  \item \(\ds T_3\,V_4^{\upgamma-1} = T_1\,V_1^{\upgamma-1}\)
\end{itemize}

De esto último se deduce que

\[
  \begin{derivation}
      \res{\left(\frac{V_1}{V_4}\right)^{\upgamma-1} =
            \left(\frac{V_2}{V_3}\right)^{\upgamma-1}}\\
    \equiv\\
      \res{ \frac{V_1}{V_4} = \frac{V_2}{V_3}}
  \end{derivation}
\]

Reemplazando en $Q_{3,4}$, y de ahí en la eficiencia, se tiene que:

\[e_c = 1 - \frac{T_3}{T_1}\]

La razón por la que esta es la máxima eficiencia posible se puede ver
asumiendo que se tiene una máquina térmica con una eficiencia $e > e_c$,
la cual alimenta una bomba térmica que vendría a ser una máquina de Carnot
operando en sentido contrario al presentado. Esto resultaría en una nueva
bomba térmica, la cual consiste en el conjunto de la máquina y la BT. El
problema radica en que la suposición de una eficiencia mayor a $e_c$,
resulta en que esta nueva bomba térmica, estaría absorbiendo energía
de un reservorio a menor temperatura que el reservorio al que le está
cediendo energía, todo esto sin recibir trabajo externo.