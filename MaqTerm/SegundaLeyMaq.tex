\section{Segunda Ley de la Termodinámica}

La segunda ley se puede expresar, desde el punto de vista de las
máquinas térmicas de la siguiente manera:

No es posible construir o encontrar una máquina térmica la cual absorba
una cantidad de energía tipo calor $Q_A$ y la transforme en trabajo
mecánico útil completamente. Es decir, toda máquina térmica expulsa
tiene una eficiencia $e < 1$.

\[T_{r,u} < Q_A \land Q_B > 0\]

Todas las cantidades corresponen a las mencionadas en la introducción.

Dicho en otras palabras, todo proceso que se realize en el universo,
aumenta la cantidad de posibles estados a los que puedan evolucionar los
sistemas involucrados. Esto último es a veces llamado ``desorden''. De
ahí, que esta ley se exprese en ocaciones como que el desorden de todo
sistema siempre está en aumento, en especial del universo entero.
Desde el punto de vista de las máquinas térmicas, si $Q_B = 0$, este
desorden disminuiría, lo cual entra en contradicción con la segunda ley.