\section{El Ciclo Otto}

Un ejemplo de una máquina térmica es un motor de combustión, en especial
el motor de ciclo Otto. Este ciclo es bien conocido por ser el usado
en la mayoría de los motores de los vehículos actualmente.

Tomando $V_1$ como el volumen de la cámara cuando el pistón está en su
punto más bajo, $V_2$ el volumen de la cámara cuando el pistón está en
su punto más alto, $P_1$ la presión del ambiente y $P_2$ la presión
en la cámara cuando el pistón comprime por completo al combustible.

\begin{enumerate}
  \item En el momento en que la válvula de admisión se abre, permitiendo la
        de combustible, la presión es la misma del ambiente ($P_1$), y el
        volumen de la cámara pasa de $V_2$ a $V_1$.

  \item Una vez se cierra la válvula de admisión, el pistón empieza a subir,
        reduciendo el volumen en la cámara. En este momento el combustible
        se empieza a comprimir. Esto se puede aproximar a un proceso
        adiabático, pues se puede considerar que no hay transferencias
        de energía tipo calor, introduciendo una presión distinta a $1$ o $2$
        a volumen $V_2$.

  \item Una vez comprimido, se genera una chispa, lo que genera a su vez
        una reacción con el combustible y una subida de presión súbita.
        Debido a la rapidez de la reacción, esto se puede aproximar a un
        proceso isométrico, con volumen $V_2$.

  \item La reacción hace que el pistón baje. Nuevamente, se considera que no
        hay transferencias de calor durante el movimiento del pistón, por lo
        que se aproxima a otro proceso adiabático, el cual finaliza a otra
        presión, con volumen $V_1$.

  \item La válvula de escape se abre, generando una caída en la presión.
        Esto se aproxima a un proceso isométrico, a volumen $V_1$.

  \item Por último, el pistón sube, liberando el gas a presión ambiente.
        Esto es un proceso isobárico a presión $P_1$, de $V_1$ a $V_2$.
\end{enumerate}

Considerando $Q_B$ el calor absorbido por la reacción de la chispa, y
$Q_A$, el calor liberado tras el espape.

\begin{center}
  \begin{tikzpicture}
    \begin{axis}[
      axis lines=center,
      xmin=-0.5,
      xmax=3,
      ymin=-1,
      ymax=7,
      xlabel=$V$,
      ylabel=$P$,
      xtick=\empty,
      ytick=\empty,
    ]
    % ~~~~ Curvas isotérmicas ~~~~~%
    \addplot[
      domain=0.8:2.5,
      samples=300,
      color=azulito,
      opacity=0.5
    ]{
      6/x
    } node[pos=0.7,yshift=12] {$T_A$};

    \addplot[
      domain=0.8:2.5,
      samples=300,
      color=azulito,
      opacity=0.5
    ]{
      1/x
    } node[pos=-0.15,yshift=-4] {$T_B$};
    % ~~~~ Procesos adiabáticos ~~~~ %
    \addplot[
      domain=2:1,
      samples=200,
      color=naranjita,
      postaction = {decorate},
      decoration = {
                    markings
                    , mark = at position 0.7 with {\arrow{Stealth[length = 1.5mm, bend]}}
                }
    ]{
      2/x^2
    } node[pos=0.6,yshift=5] {\resizebox{1.25mm}{1.25mm}{2}};

    \addplot[
      domain=1:2,
      samples=500,
      color=naranjita,
      postaction = {decorate},
      decoration = {
                    markings
                    , mark = at position 0.7 with {\arrow{Stealth[length = 1.5mm, bend]}}
                }
    ]{
      6/x^2
    } node[pos=0.7,yshift=6] {\resizebox{1.25mm}{1.25mm}{4}};

    % ~~~~ Procesos isométricos ~~~~ %
    \draw[
      postaction = {decorate},
      decoration = {
                    markings
                    , mark = at position 0.5 with {\arrow{Stealth[length = 1.5mm, bend]}}
                }
    ] (1,{2/1^2}) -- (1,{6/1^2})
      node[pos=0.5,xshift=-6] {\resizebox{1.25mm}{1.25mm}{3}};
    \draw[
      postaction = {decorate},
      decoration = {
                    markings
                    , mark = at position 0.5 with {\arrow{Stealth[length = 1.5mm, bend]}}
                }
    ] (2,{6/2^2}) -- (2,{2/2^2})
      node[pos=0.5,xshift=6] {\resizebox{1.25mm}{1.25mm}{5}};

    % ~~~~ Procesos isobáricos ~~~~ %
    \draw[
      postaction = {decorate},
      decoration = {
                    markings
                    , mark = at position 0.3 with {\arrow{Stealth[length = 1.5mm, bend]}}
                }
    ] (1,{2/2^2}) -- (2,{2/2^2})
      node[pos=0.3, yshift=4.5] {\resizebox{1.25mm}{1.25mm}{1}};
    \draw[
      postaction = {decorate},
      decoration = {
                    markings
                    , mark = at position 0.3 with {\arrow{Stealth[length = 1.5mm, bend]}}
                }
    ] (2,{2/2^2 - 0.15}) -- (1,{2/2^2 - 0.15})
      node[pos=0.2, yshift=-4.5] {\resizebox{1.25mm}{1.25mm}{6}};
    % ~~~~ Valores en ejes ~~~~ %
    \draw (0,{2/2^2}) -- (-0.2,{2/2^2}) node[left] {$P_1$};
    \draw (0,{6/1^2}) -- (-0.2,{6/1^2}) node[left] {$P_2$};
    \draw (1,0) -- (1,-0.2) node[below] {$V_2$};
    \draw (2,0) -- (2,-0.2) node[below] {$V_1$};

    % ~~~~ Temperaturas auxiliares ~~~~ %
    \draw[color=amarillito, ->] ({1-0.05},{2/1^2}) to[bend left] (0.7,3)
      node[above] {$T_A'$};
    \draw[color=amarillito, ->] ({2},{6/2^2+0.05}) to[bend left] (2.3,2.1)
      node[right] {$T_B'$};
    \end{axis}
  \end{tikzpicture}
\end{center}

La eficiencia entonces es:

\[e_{\text{Otto}} = 1 - \frac{Q_B}{Q_A}\]

\begin{itemize}
  \item De (II) se tiene que: $P_1\,V_1^\upgamma = \upalpha_1$, con la ley de
  los gases ideales, se puede deducir que $T\,V^\upgamma$ es constante en
  estos procesos. De ahí, $T_B' = T_B \left(\frac{V_1}{V_2}\right)^{\upgamma-1}$.

  \item De forma similar, en (IV), se obtiene que
  $T_A' = T_A \left(\frac{V_2}{V_1}\right)^{\upgamma-1}$.

  \item De (III) se tiene que $Q_A = n\,C_V\,(T_A - T_B')$.
  
  \item De (V) se tiene que $Q_B = n\,C_V\,(T_A' - T_B)$.
\end{itemize}

Reemplazando $T_A'$ y $T_B'$ por lo obtenido de (II) y (IV), y computando
el cociente de las energías de tipo calor para obtener la eficiencia,
se obtiene que:

\[e_{\text{Otto}} = 1 - \left(\frac{V_2}{V_1}\right)^{1 - \upgamma}\]

Por ejemplo, un motor V12 a gasolina con \qty{5204}{\cm^3}. Cada pistón entonces
tendrá \qty{433.667}{\cm^3}. Si su relación de compresión ($V_1/V_2$)
es de $9.3$, entonces, su eficiencia será $1 - 9.3^{1 - 1.4} = 0.59$ que
corresponde al $59\%$
