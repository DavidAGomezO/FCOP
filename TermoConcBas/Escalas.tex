\section{Escalas de Temperatura}

Las escalas de temperatura no son más que las diferentes unidades de
medida para la \\temperatura. Para ejemplificar la relación entre las más
conocidas se tomarán dos puntos: la trancisión de fase de fusión y
ebullición del agua.
\RenewDocumentCommand{\degree}{O{C}}{{}^{\circ} \text{#1}}
\begin{center}
    \begin{tabular}{c|c|c|c}
        Trancisión de fase  & Celsius ($\degree$)   & 
        Farenheit ($\degree[F]$) & Kelvin ($K$)\\
        \hline
        Fusión              & 0                     & 32    & 273.15\\
        Ebullición          & 100                   & 212   & 373.15
    \end{tabular}
\end{center}

Se puede ver que tanto la escala Celsius como la Kelvin tienen $100$
unidades entre las dos transiciones de fase, mientras que la Farenheit
tiene $180$. Como estas escalas mantienen una relación lineal con la
energía de tipo calor, la conversión se vuelve un problema simple de
proporcionalidad:

\begin{center}
    \renewcommand{\arraystretch}{1.75}
    \begin{tabular}{|c|c|c|c|}
        \hline
        \multicolumn{4}{|c|}{Conversión entre escalas (columna $\to$ fila)}\\
        \hline
        Conversión 
            & Celsius ($\degree$)
            & Farenheit ($\degree[F]$)
            & Kelvin ($K$)\\
        \hline
        Celsius ($\degree$)
            & $\degree=\degree$
            & $\degree = \dfrac{100}{180}(\degree[F]-32)$
            & $\degree = (K) - 273.15$\\
        \hline
        Farenheit ($\degree[F]$)
            & $\degree[F] = \dfrac{180}{100}\degree + 32$
            & $\degree[F] = \degree[F]$
            & $\degree[F] = \dfrac{180}{100}(K)-459.67$\\
        \hline
        Kelvin ($K$)
            & $(K) = \degree + 273.15$
            & $(K) = \dfrac{100}{180}(\degree[F] + 459.67)$
            & $(K) = (K)$\\
        \hline
    \end{tabular}
\end{center}

\section{Ejemplo Numérico}

Pasar a las otras dos escalas las siguientes temperaturas:
$3\degree\,,\,14\degree\,,\,15\degree\,,\,92\degree\,,\,65\degree$

\begin{center}
    \begin{tabular}{c|c}
        Farenheit           & Kelvin\\
        \hline
        $37.4\degree[F]$    & $276.15$\\
        $57.2\degree[F]$    & $287.15$\\
        $59\degree[F]$      & $288.15$\\
        $197.6\degree[F]$   & $365.15$\\
        $149\degree[F]$     & $338.15$
    \end{tabular}
\end{center}